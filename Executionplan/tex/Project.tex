\documentclass[11pt]{report}
\begin{document}


\section{Project description}
The machines used for production in factories today range in the hundreds of thousands of euros. These machines are highly specialized for specific tasks and are usually not adaptable to their environment or needs. This creates a very inefficient setup for the manufacturers. Because of this, the initial project was created to come up with a cheap and reusable alternative to modern day factory machines. The resulting idea from that project was to create a grid of machines, each capable of a task depending on which modules are attached to the machine. These machines are called equiplets. This way of having a manufacturing floor is called ‘Agile manufacturing'\cite{IAT2013}.

One of the major drawbacks of current machines is that they are suited to only a single specific task. Therefore one of the aspects of this project includes reconfigurability for the equiplets. The idea is to be able to attach and detach modules from the system without first needing to shut it down. This means the system will detect changes in its modules and adapt to those changes. This in turn means that the production machine can be reconfigured on-the-fly to allow for the creation of new products without a long period of downtime.

The focus of agile manufacturing is to create a manufacturing grid that can produce a large variety of products with a relatively low amount. Also the goal is to make relatively cheap manufacturing robots ( equiplets ) and modules.

To successfully create such a platform there has to be a solid architecture. As stated in the paper about this question\cite{MASTA2013}, a hybrid architecture consisting of ROS and JADE can be the solution.
This new platform in development is called REXOS. The research that will be done as described in this document will be beneficial to the REXOS platform.

\section{Project organization}
%de projectorganisatie;
This research project consists of 3 team members and one project 
supervisor. The research project has close relations with other projects 
beneficial to REXOS. There is a team of interns that is developing the 
vision part (vision team). There is another research team researching 
the logistics part of REXOS. The research team conducting the research 
described in this document has close collaboration with the vision team, 
because the two teams are using the same hardware ( the equiplet ). The 
people involved in the research project are:

\begin{table}[htbp]

\centering
\normalsize
\begin{tabular}{|l|l|}
\hline
Name & Role\\
\hline
E. Puik & Founding father\\
\hline
D. Telgen & Project supervisor\\
\hline
A. Streng & Team supervisor, researcher/software developer and scrum master\\
\hline
D. Jenkins & researcher/software developer\\
\hline
R. Scheefhals & researcher/software developer\\
\hline
A. Hustinx & Team supervisor, intern/software developer and scrum master\\
\hline
T. Bakker & intern/software developer\\
\hline
G. Hakopian & intern/software developer\\
\hline
\end{tabular}
\caption{Project team}
\end{table}
The researchers involved in the project are students at the university 
in Utrecht. They work directly under D. Telgen in a researching capacity.

\subsection{Scrum}
As a projectmanagement method \emph{Scrum}\cite{SCRUM}
will be used. Scrum is an \emph{Agile projectmanagement method}. In 
Scrum, 'Agile' means that the enduser or project supervizor can change 
certain aspect of the project while the project is in progress. The 
Scrum method supports this kind of workflow.

The work that has to be done are called stories in Scrum. These stories 
are short worktasks that mostly can be finished within one day. These 
stories are stored in a so called \emph{Product Backlog}. These stories 
will be given an estimation of how long it will take to finish the story.

In Scrum, project time is being divided in iterations of fixed lengths. 
These iterations are called \emph{sprints}. In this project sprints will 
be planned of a length of one week. When a sprint is being planned, 
stories from the Product Backlog will be picked. which stories are being 
picked depends on what the goal is of the sprint and how much work can 
be done by the team. Stories that could not be completed in that sprint 
will be put back into the product backlog so they can be planned into a 
new sprint.

In this project, the digital Scrumboard 
\emph{Assembla}\footnote{\url{https://www.assembla.com/‎}} will be used. 
Assembla is a Scrum tool that provides a good overview of all Scrum 
functionality.

Considering this project is a research project that can change a lot, 
Scrum will be the best method.

\subsection{Communication}

In this project Daniel Telgen is the project supervisor. While 
conducting the research, all communication about the research being done 
goes through the project supervisor and there will be weekly meetings. 
In these meetings the progress will be discussed and new assignments 
will be planned for the next week in the form of sprints. The project 
leader wil also be available daily through mail.

A time sheet will be made during the project to make sure that each team 
member is present. This has to be done since some members need a day off 
every week for other obligations.

\section{Current state of the project}
Up until this moment the development of project REXOS was focused on the initial architecture and functionality of the equiplets. In this stage of the project the focus is drawn towards what the current performance of REXOS is and how reconfigurability can be implemented in REXOS. At this moment only one equiplet run the required software and contains the hardware to test performance. This means that data regarding the performance of multiple equiplets cannot be gathered, however, more equiplets are expected to arrive soon. Until these extra equiplets arrive, meaningful test data cannot be gathered. Initially there was some difficulty getting the platform to run in a stable manner. The problems causing this have been solved however, which now allows for proper testing to be conducted.

A system for gathering testing data has been partially implemented into the platform. This system gathers information about the communication between agents on the platform. Communication between other parts of the platform is not implemented yet however. This implementation will need to be completed before performance testing starts.
 
Reconfigurability has not been implemented at all, although the system does use MAST.

\section{Project activities}
%de projectactiviteiten met een beschrijving van de onderlinge samenhang, de 
%mijlpalen en een fasering in de tijd met een schatting van de te besteden uren voor 
%de verschillende uit te voeren activiteiten

During this project we will mainly focus on three major activities. Gaining a \textbf{stable platform}, gathering a wide variaty of \textbf{performance metrics} and implementing / researching \textbf{reconfigurability}. The stable platform is needed to continue working on this project. Not only for the current group of researchers / interns working on the project, but for the next generation of students who will participate in this project aswell. The performance metrics are of great importance because they are needed for an extension of the FAIM\cite{FAIM2013} paper in a journal. Reconfigurability is the main focus of our research project. By researching (and possibly implementing) reconfigurability, the project would get a major boost. It is also a core feature of the current system to have reconfigurability at its disposal.

	\subsection{Phases}
Due to a upcoming deadline for the precision convention\footnote{http://www.precisiebeurs.nl/} there are two different phases that are slightly parallel to each other. The main phase, in which reconfigurability is planned, is defined on the wiki\footnote{\url{http://wiki.agilemanufacturing.nl/index.php/REX_3.0_Phase_1}}. The other phase, which is defined because of the precision convention, is mainly focusing on gaining a stable platform and writing some demonstrators.
After both those phases there is one more phase in which we will finish implementing reconfiguration.
	
	\begin{itemize}
    	\item Performance phase\\
    		This main phase will focus on writing code to get the performance metrics. Keeping the second goal of creating a stable system in mind, some work on the stability of the platform will be done. In order to get these performance metrics, specific testing code has to be written and researched. A means to gather this data (e.c. logging) has to be implemented as well. 
    		
    	\item Precision phase\\
    		The phase leading up to the precision convention will mainly be about stabilizing the platform and writing demonstrator code. This phase is almost completely parallel to the Reconfigurability phase. Due to the upcoming precision convention, a total of 3 demonstrators have to be written. The whole platform itself, including making of a product, planning and producing of the product has to work as well.
    		
    	\item Reconfigurability phase\\
    		This phase will be all reconfiguring. This phase starts after the precision fair and will last untill the end of the research semester. Due to the nature of the project reconfiguring is an important aspect. During the first two phases the interns will have done most work on implementing reconfigurability on a lower level thus allowing research to be done on the intelligence behind reconfigurability. This means researching the best way to deal with changing modules and hardware on the MAS level.
	\end{itemize}
	
	
	\subsection{Milestones}
	Throughout the project several important milestones are defined. It is clear that all of these need to be achieved in order for our research to be succesfull (maybe with exception of the demonstrators, but these are quite important to the HU).
		\begin{itemize}
		\item \textbf{Stable platform}\\
		In order for this project to work, the platform on which it is designed needs to be stable. This is currently not the case. Most of the system crashes whenever unsuspected conditions are present and there are still features that need to be implemented. The most important features are:
		\begin{itemize}
    	\item \emph{Handle multiple equiplets}\\
    	Currently the platform isn't capable of handling more then one equiplet. As soon as multiple equiplets are present within the system, it overloads\footnote{concluded after several tests}. Part of the solution is to test and search for the cause of this problem  and solve it.\\
    	\item \emph{Handle multiple product agents}\\
    	Whenever multiple product agents are present within the grid, the planning of product steps tends to mismatch. Research is needed as to why this is happening, and a proper solution has to be implemented.\\
		\end{itemize}
    	\item \textbf{Performance metrics}\\
    	Due to an upcoming deadline for a journal, there is a need to know certain metrics of the (stable) system. These need to be gathered and parsed into something usable. Determining which metrics should be gathered has been done and only implementing the code to do so remains. These metrics are:
		\begin{itemize}
    	\item \emph{The amount of JADE messages sent}
		\item \emph{JADE messages that have been lost / not received}
		\item \emph{number of timeouts between JADE message communications}
		\item \emph{ROS messages are being sent through an actionserver}
		\item \emph{The IO to and from the MongoDB blackboards. Fetch speed/ process time / tailable cursor speeds}
		\end{itemize}
		In order to gather sensible data, test scenario's will have to be constructed. Code needs to be implemented as well to gather the data, and process it in usefull statistics. 
    	\item \textbf{Reconfigurability}\\
    	Reconfigurability is an important aspect of the platform\cite{Koren} but not yet implemented. Once reconfigurability on a modular level has been achieved\footnote{This is done by the interns}, the reconfigurability on the MAS level has be researched and implemented. This means doing research on how to make the software agents 'aware' of their modules, and how to let them act when one changes. The intelligence should be able to deal with unsuspected conditions, malfunctions and human errors.
    	\item \textbf{Demo's}\\
    	Due to the upcoming precision convention, demonstrators need to work and the platform has to be stable. During this convention, the goal is to show a demo of the completely functional system.
    	The setup on the convention will be as following:
		\begin{itemize}
    	\item \emph{Two pick and place robots}
    	\item \emph{3d printer}
    	\item \emph{Stewart platform delta robot}
		\end{itemize}
    	All these equiplets will be placed within the same grid, and will operate on the REXOS platform. This means that the platform should be fully functional at this time. Demonstrator code needs to be written as well to guarantee smooth demo's.
		\end{itemize}
	
	\subsection{Planning}
	Due to the abstract nature and timing of the phases, the planning will have some parallel elements. Details of the planning will be done "on demand" depending on Customer demand. According to the SCRUM method.\\
	\begin{table}[htbp]
	\begin{tabular}{ | l | l |  p{8cm} |}
		\hline
		{\bf Phase} & {\bf Date span} & {\bf Description} \\ \hline
		performance phase & 21-08-2013 - 03-12-2013 & Stabilize platform, gain performance metrics \\ \hline
		precision phase & 29-09-2013 - 03-12-2013 & performance metrics and stabilize platform. Write demonstrators. \\ \hline
		reconfigurability phase & 03-12-2013 - 01-02-2014 & research and implement reconfigurability \\ \hline
\end{tabular}
\caption{Project phases}
\end{table}

\section{Methods and tools}
\label{sec:mat}
\subsection{Jade}
Java Agent DEvelopment Framework is a software framework to develop distributed agent-based applications in compliance with the FIPA\footnote{http://www.fipa.org} specifications for interoperable intelligent multi-agent systems. An application based on JADE is composed of a set of components called "agents" implementing the pieces of functionality required by the application. JADE primarily provides the Agent and Behaviour (a task to be executed by an agent) abstractions, transparent distribution of agents accross a wide range of devices, peer-to-peer communication between agents and a publish-subscribe discovery mechanisms that allows agents finding each other. Furthermore JADE provides a number of additional features such as agent mobility, ontologies and content language support, fault tolerance and web services integration and a rich suite of graphical tools that facilitate the administration of a JADE based application. JADE provides:
\begin{itemize}
 \item An environment where JADE agents are executed.
 \item Class Libraries to create agents using heritage and redefinition of behaviors.
 \item A graphical toolkit to monitoring and managing the platform of Intelligent Agent agents.
\end{itemize}

\subsection{ROS}
ROS is the main platform used to create and run code for modules (called nodes in ROS). ROS is an multi-langual, peer-to-peer, tool-based, thin and open source framework for robotics software\cite{ROS}. ROS can be used as an abstraction layer for the hardware in modular machine and robots. ROS utilizes the C++ program language (as well as 3 others\cite{ROS}) and can be used directly to control hardware systems in realtime.

\subsection{MongoDB}
MongoDB is the datacommunication layer used to communicate between the MAS and Ros layer. MongoDB is a no-SQL database, which means that it operates with a flexible schema. From the mongo website: "\emph{documents in the same collection do not need to have the same set of fields or structure, and common fields in a collection’s documents may hold different types of data.}"\footnote{http://docs.mongodb.org/manual/core/data-modeling/}

\section{Literature}
%een beschrijving van de belangrijkste literatuur die onderzocht zal worden 
%(optioneel);
The majority of the literature will consist of the papers and thesis' that Daniel Telgen and Leo van Moergestel have written themselves. All the papers in this context are accepted at international conferences\footnote{Most of them. If they where rejected this will be noted.}. These papers are available on google drive.
\clearpage
\section{Risks}
%de eventuele risico's met maatregelen;


\begin{table}[htbp]
\begin{tabular}{|l|l| p{5cm}| p{6cm}|}
\hline
{\bf Severity} & {\bf Chance} & {\bf Risk} & {\bf Solution}\\
\hline
10 & 3 & Because there are a lot of bugs to fix in REXOS it is possible there will not be enough time left for the actual research & This can be avoided by keeping track of upcoming deadlines and prioritising bugs that can be fixed in the time alotted\\
\hline
3 & 6 & When running performance tests on a system that is not yet fully stable, it is possible faulty data will be gathered and the results will be inconsistent & To limit this, it is useful to know about every aspect in REXOS so a faulty process can be identified earlier\\
\hline
5 & 2 & Failing server with loss of documents / wiki entries & Keep an off-site backup of all the data and having a backup server available\\
\hline
8 & 4 & Not being able to get sensible conclusions out of the performance test results & Make sure that the results are reproducable and the test data that is being gathered is sensible. Also keep an eye on what we want to accomplish with gathered data. If the gathered data has no meaning afterwards, we dont need it.\\
\hline
5 & 3 & Having an unstable JADE platform & When stabilizing the REXOS platform, keep testing with different data and different cases. While improving the platform, keep an eye on how JADE responds to it and this should not happen \\
\hline
5 & 3 &MongoDB not being the right engine for blackboards on a large scaled platform, i.e. it is too slow & When this happens, there will be two options. One is to look into the problem if the performance can be improved. The other is to look into alternatives of MongoDB.\\
\hline
10 & 1 &ROS not being able to perform on a large scaled platform & When this happens, the project has to be redesigned, or a different machine control environment has to be chosen. But it is very unlikely this will happen since ROS is designed to be scalable\cite{ROS}.\\
\hline

\end{tabular}
\caption{Project risks}
\end{table}

\end{document}