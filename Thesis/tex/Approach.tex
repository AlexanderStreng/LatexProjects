\documentclass{../local}
\begin{document}

\section{Planning - TODO}
	Due to the abstract nature and timing of the phases, the planning will have some parallel elements. Details of the planning will be done "on demand" depending on customer demand. According to the SCRUM method.\\
\begin{table}[htbp]
	\begin{tabular}{ | l | l |  p{8cm} |}
		\hline
		{\bf Phase} & {\bf Date span} & {\bf Description} \\ \hline
		performance phase & 21-08-2013 - 03-12-2013 & Stabilize platform, gain performance metrics \\ \hline
		precision phase & 29-09-2013 - 03-12-2013 & performance metrics and stabilize platform. Write demonstrators. \\ \hline
		reconfigurability phase & 03-12-2013 - 01-02-2014 & research and implement reconfigurability \\ \hline
\end{tabular}
\caption{Project phases}
\end{table}

\section{Communication - TODO}
<% === Text is future tense, needs to be changed to past tense >
In this project Daniel Telgen is the project supervisor. While 
conducting the research, all communication about the research being done 
goes through the project supervisor and there will be weekly meetings. 
In these meetings the progress will be discussed and new assignments 
will be planned for the next week in the form of sprints. The project 
leader wil also be available daily through mail.

A time sheet will be made during the project to make sure that each team 
member is present. This has to be done since some members need a day off 
every week for other obligations.

\section{Methods \& tools}
This section will cover the methods and tools used during the project. We used a wide arrange of tools \& methods and the most important ones are described below.

\subsection{Scrum - TODO}
<% === Text is future tense, needs to be changed to past tense >
As a projectmanagement method \emph{Scrum}\cite{SCRUM}
will be used. Scrum is an \emph{Agile projectmanagement method}. In 
Scrum, 'Agile' means that the enduser or project supervizor can change 
certain aspect of the project while the project is in progress. The 
Scrum method supports this kind of workflow.

The work that has to be done are called stories in Scrum. These stories 
are short worktasks that mostly can be finished within one day. These 
stories are stored in a so called \emph{Product Backlog}. These stories 
will be given an estimation of how long it will take to finish the story.

In Scrum, project time is being divided in iterations of fixed lengths. 
These iterations are called \emph{sprints}. In this project sprints will 
be planned of a length of one week. When a sprint is being planned, 
stories from the Product Backlog will be picked. which stories are being 
picked depends on what the goal is of the sprint and how much work can 
be done by the team. Stories that could not be completed in that sprint 
will be put back into the product backlog so they can be planned into a 
new sprint.

In this project, the digital Scrumboard 
\emph{Assembla}\footnote{https://www.assembla.com/‎} will be used. 
Assembla is a Scrum tool that provides a good overview of all Scrum 
functionality.

Considering this project is a research project that can change a lot, 
Scrum will be the best method.

\subsection{Eclipse - TODO}
Eclipse is a community for individuals and organizations who wish to collaborate on commercially-friendly open source software. Its projects are focused on building an open development platform comprised of extensible frameworks, tools and runtimes for building, deploying and managing software across the lifecycle.
The Eclipse project is an integrated development environment (IDE). Written mostly in Java, Eclipse can be used to develop applications in Java. By mean of various plug-ins, Eclipse maybe also be used to develop applications in other programming languages: Ada, C, C++, COBOL, Fortran, Haskell JavaScript, Lasso, Perl, PHP, Python, R, Ruby, Scala, Clojure, Groovy, Scheme and Erlang. It can also be used to develop packages for the software Mathematica.

\subsection{ROS - TODO}
ROS is the main platform used to create and run code for modules (called nodes in ROS). ROS is an multi-langual, peer-to-peer, tool-based, thin and open source framework for robotics software\cite{ROS}. ROS can be used as an abstraction layer for the hardware in modular machine and robots. ROS utilizes the C++ program language (as well as 3 others\cite{ROS}) and can be used directly to control hardware systems in realtime.

\label{sec:mat}
\subsection{Jade - TODO}
Java Agent DEvelopment Framework is a software framework to develop distributed agent-based applications in compliance with the FIPA\footnote{http://www.fipa.org} specifications for interoperable intelligent multi-agent systems. An application based on JADE is composed of a set of components called "agents" implementing the pieces of functionality required by the application. JADE primarily provides the Agent and Behaviour (a task to be executed by an agent) abstractions, transparent distribution of agents accross a wide range of devices, peer-to-peer communication between agents and a publish-subscribe discovery mechanisms that allows agents finding each other. Furthermore JADE provides a number of additional features such as agent mobility, ontologies and content language support, fault tolerance and web services integration and a rich suite of graphical tools that facilitate the administration of a JADE based application. JADE provides:
\begin{itemize}
 \item An environment where JADE agents are executed.
 \item Class Libraries to create agents using heritage and redefinition of behaviors.
 \item A graphical toolkit to monitoring and managing the platform of Intelligent Agent agents.
\end{itemize}

\subsection{MongoDB - TODO}
MongoDB is the datacommunication layer used to communicate between the MAS and ROS layer. MongoDB is a no-SQL database, which means that it operates with a flexible schema. From the mongo website: "\emph{documents in the same collection do not need to have the same set of fields or structure, and common fields in a collection’s documents may hold different types of data.}"\footnote{http://docs.mongodb.org/manual/core/data-modeling/}

\section{Feedback \& Brainstorms}

\section{Implementation}

\section{Evaluation}

\end{document}

