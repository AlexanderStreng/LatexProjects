\documentclass{../local}
\begin{document}

\section{Planning}
	During the initialisation of the project, three phases where determined. In these phases certain deadlines where set. These phases where used as milestones instead of an accurate planning. Detailed planning is done by using SCRUM.
\begin{table}[htbp]
	\begin{tabular}{ | l | l |  p{5.7cm} |}
		\hline
		{\bf Phase} & {\bf Date span} & {\bf Description} \\ \hline
		performance phase & 21-08-2013 - 03-12-2013 & Stabilize platform, gain performance metrics \\ \hline
		precision phase & 29-09-2013 - 03-12-2013 & performance metrics and stabilize platform. Write demonstrators. \\ \hline
		reconfigurability phase & 03-12-2013 - 01-02-2014 & research and implement reconfigurability \\ \hline
\end{tabular}
\caption{Project phases}
\end{table}
Not all these phases remained intact througout the project. No reconfiguring was done during the reconfiguring phase.


\section{Communication}
 While conducting research, all communication about the research went through the project supervisor and there were meetings whenever needed each week. In these meetings the progress was discussed and new assignments were started and planned for the upcoming days in the form of sprints. The project supervisor was available at almost any other time through e-mail.

During the project a time sheet was created to record the attendance of each team member.

\section{Methods \& Tools}
This section will cover the methods and tools used during the project. A wide range of tools \& methods was used. The most important ones are in the following sections.

\label{sec:mat}
\subsection*{Jade}
Jade is short for Java Agent DEvelopment Framework\footnote{http://jade.tilab.com/}. Using this framework agent based applications can be written in Java, taking advantage of the Agent and Behaviour abstractions provided. Jade also provides features such as agent mobility, ontologies and content language support and fault tolerance. The framework is FIPA\footnote{http://www.fipa.org} compliant.

Provided with Jade is:
\begin{itemize}
 \item An environment where JADE agents are executed.
 \item Class Libraries to create agents using heritage and redefinition of behaviors.
 \item A graphical toolkit to monitoring and managing the platform of Intelligent Agent agents.
\end{itemize}

The features mainly used during the project were the agent and behaviour abstractions and ontologies.

\subsection*{ROS}
ROS is the main platform used to create and run code for modules (called nodes in ROS). ROS is an multi-lingual, peer-to-peer, tool-based, thin and open source framework for robotics software\cite{ROS}. ROS can be used as an abstraction layer for the hardware in modular machine and robots. ROS utilizes the C++ program language (as well as three others\cite{ROS}) and can be used to directly control hardware systems in realtime.

\subsection*{MongoDB}
MongoDB is used as the datacommunication layer used to communicate between the MAS and ROS layer. MongoDB is a no-SQL database, which means that it operates with a flexible schema. From the mongo website: "\emph{documents in the same collection do not need to have the same set of fields or structure, and common fields in a collection’s documents may hold different types of data.}"\footnote{http://docs.mongodb.org/manual/core/data-modeling/}

\subsection*{Apache Ant}
Apache Ant is a Java library and command-line tool used for building Java applications through the use of build files\footnote{http://ant.apache.org/}. Ant can also be used to build non Java applications, such as C or C++ applications. For this reason, Ant was used in the project to build both the MAS layer, which is written in Java, as well as the ROS layer, which is written in C++. Ant provides built-in tasks making the compiling, assembling, testing and running of applications simple and fast.

\subsection*{Scrum}
\emph{Scrum} is an \emph{Agile projectmanagement method}\cite{SCRUM}. In Scrum, 'Agile' means that the enduser or project supervizor can change certain aspect of the project while the project is in progress. The Scrum method supports this kind of workflow. 
The work that has to be done are called stories in Scrum. These stories are short worktasks that mostly can be finished within one day. These stories are stored in a so called \emph{Product Backlog}. These stories will be given an estimation of how long it will take to finish the story.

In Scrum, project time is being divided in iterations of fixed lengths. These iterations are called \emph{sprints}. In this project sprints will be planned of a length of one week. When a sprint is being planned, 
stories from the Product Backlog will be picked. which stories are being picked depends on what the goal is of the sprint and how much work can be done by the team. Stories that could not be completed in that sprint 
will be put back into the product backlog so they can be planned into a new sprint.

In this project, the digital Scrumboard 
\emph{Assembla}\footnote{https://www.assembla.com/‎} was used. Assembla is a platform that provides a set of tools for the Scrum method.

\begin{figure}[h!]
	\centering
	\includegraphics[width=10cm]{../images/assembla-example.jpg}
	\caption{Assembla}
\end{figure}

\subsection*{Programming Conventions}
Since multiple programmers will be working on this project, standardized means of development are needed. The project team has set up programming conventions that will be used by all of the developers. All of the different aspects of programming have been described and standardized and can be found on the wiki\footnote{http://wiki.agilemanufacturing.nl/index.php/Programming\_conventions}.

\subsection*{Eclipse}
Eclipse is an integrated development environment (IDE)\footnote{http://www.eclipse.org/}. The IDE provides an easy to use interface for programming with the Java language. By means of various plug-ins, Eclipse may also be used to develop applications in other programming languages such as: C, C++, PHP and Python. When developing Java applications, eclipse also supports debugging of the application.

\subsection*{Google Drive}
Google Drive is a file storage and synchronization service provided by Google\footnote{http://en.wikipedia.org/wiki/Google\_Drive}, released on April 24, 2012, which enables user cloud storage, file sharing and collaborative editing. Files shared publicly on Google Drive can be searched with web search engines. Google Drive is the home of Google Docs, an office suite of productivity applications, that offer collaborative editing on documents, spreadsheets, presentations, and more.

\subsection*{Wiki}
Over the course of the REXOS project, the knowledge base expanded. In this project the decision was made to create an wiki\footnote{http://wiki.agilemanufacturing.nl}. Before this wiki started, all of the knowledge resided in final report or transfer documents on Google Drive. This became cluttered, because these documents contained duplicate and redundant data. Not all documents have been transferred from Google Drive to the wiki yet. 

\subsection*{Version Control}
In this project git was used as version control software. "\emph{Git is a free and open source distributed version control system designed to handle everything from small to very large projects with speed and efficiency. Git is easy to learn and has a tiny footprint with lightning fast performance. It outclasses SCM tools like Subversion, CVS, Perforce, and ClearCase with features like cheap local branching, convenient staging areas, and multiple workflows.}"\footnote{http://git-scm.com/}

GitHub\footnote{http://www.github.com/} was used to host the source code for the project. GitHub was founded in 2008 for the purpose of simplifying the sharing of code.

\begin{figure}[h!]
	\centering
	\includegraphics[width=10cm]{../images/github-example.jpg}
	\caption{GitHub}
\end{figure}

\section{Feedback \& Brainstorms}

\subsection*{Feedback}
Throughout the research project, feedback was provided by the project supervisor. This was generally supplied in the form of email, physical contact and through the use of a message board. One exception being a performance review halfway the semester. Whenever a member of the team required input or feedback on something, that member would send an email to the project supervisor or post a message on the message board. The project supervisor would reply to these requests with feedback.

At the end of the project, a peer review is to be held in which the team members will give each other feedback on their performance during the project.

\subsubsection*{Performance review}
Halfway through the research semester, each team member had a performance review with the project supervisor in which the supervisor gave feedback to each member about their personal performance.

\subsection*{Brainstorms}
When necessary, brainstorm sessions were held. These brainstorms generally consisted of the project supervisor, the researchers and additionally another small team of interns. Whenever the team and the project supervisor agreed that a brainstorm session was needed a date would be chosen. In the brainstorm sessions a whiteboard was used to write down everything that was discussed. These sessions generally resulted in a lot of new ideas for the project.

\end{document}

