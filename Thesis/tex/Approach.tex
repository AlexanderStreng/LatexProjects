\documentclass{../local}
\begin{document}

\section{Planning - TODO}
	Due to the abstract nature and timing of the phases, the planning will have some parallel elements. Details of the planning will be done "on demand" depending on customer demand. According to the SCRUM method.\\
\begin{table}[htbp]
	\begin{tabular}{ | l | l |  p{8cm} |}
		\hline
		{\bf Phase} & {\bf Date span} & {\bf Description} \\ \hline
		performance phase & 21-08-2013 - 03-12-2013 & Stabilize platform, gain performance metrics \\ \hline
		precision phase & 29-09-2013 - 03-12-2013 & performance metrics and stabilize platform. Write demonstrators. \\ \hline
		reconfigurability phase & 03-12-2013 - 01-02-2014 & research and implement reconfigurability \\ \hline
\end{tabular}
\caption{Project phases}
\end{table}

\section{Communication - TODO}
In this project Daniel Telgen was (and still is) the project supervisor. While conducting the research, all communication about the research being done went through the project supervisor and there were meetings when needed each week. In these meetings the progress was discussed and new assignments were planned for the upcoming days in the form of sprints. The project supervisor was available at almost any other time through mail.

During the project a time sheet was made to record the presence of each team member. The reasoning behind this document was that certain team members required a day off each week to fulfill other obligations.

\section{Methods \& tools}
This section will cover the methods and tools used during the project. We used a wide arrange of tools \& methods and the most important ones are described below.

\subsection*{Scrum - TODO}
\emph{Scrum}\cite{SCRUM} is an \emph{Agile projectmanagement method}. In 
Scrum, 'Agile' means that the enduser or project supervizor can change 
certain aspect of the project while the project is in progress. The 
Scrum method supports this kind of workflow.

The work that has to be done are called stories in Scrum. These stories 
are short worktasks that mostly can be finished within one day. These 
stories are stored in a so called \emph{Product Backlog}. These stories 
will be given an estimation of how long it will take to finish the story.

In Scrum, project time is being divided in iterations of fixed lengths. 
These iterations are called \emph{sprints}. In this project sprints will 
be planned of a length of one week. When a sprint is being planned, 
stories from the Product Backlog will be picked. which stories are being 
picked depends on what the goal is of the sprint and how much work can 
be done by the team. Stories that could not be completed in that sprint 
will be put back into the product backlog so they can be planned into a 
new sprint.

In this project, the digital Scrumboard 
\emph{Assembla}\footnote{https://www.assembla.com/‎} was used. Assembla is a Scrum tool that provides a good overview of all Scrum functionality.

\begin{figure}[h!]
	\caption{Assembla}
	\centering
	\includegraphics[width=10cm]{../images/assembla-example.jpg}
\end{figure}

\subsection*{Eclipse - TODO}
Eclipse is a community for individuals and organizations who wish to collaborate on commercially-friendly open source software. Its projects are focused on building an open development platform comprised of extensible frameworks, tools and runtimes for building, deploying and managing software across the lifecycle.
The Eclipse project is an integrated development environment (IDE). Written mostly in Java, Eclipse can be used to develop applications in Java. By mean of various plug-ins, Eclipse maybe also be used to develop applications in other programming languages: Ada, C, C++, COBOL, Fortran, Haskell JavaScript, Lasso, Perl, PHP, Python, R, Ruby, Scala, Clojure, Groovy, Scheme and Erlang. It can also be used to develop packages for the software Mathematica.

\subsection*{ROS - TODO}
ROS is the main platform used to create and run code for modules (called nodes in ROS). ROS is an multi-langual, peer-to-peer, tool-based, thin and open source framework for robotics software\cite{ROS}. ROS can be used as an abstraction layer for the hardware in modular machine and robots. ROS utilizes the C++ program language (as well as 3 others\cite{ROS}) and can be used directly to control hardware systems in realtime.

\label{sec:mat}
\subsection*{Jade - TODO}
Java Agent DEvelopment Framework is a software framework to develop distributed agent-based applications in compliance with the FIPA\footnote{http://www.fipa.org} specifications for interoperable intelligent multi-agent systems. An application based on JADE is composed of a set of components called "agents" implementing the pieces of functionality required by the application. JADE primarily provides the Agent and Behaviour (a task to be executed by an agent) abstractions, transparent distribution of agents accross a wide range of devices, peer-to-peer communication between agents and a publish-subscribe discovery mechanisms that allows agents finding each other. Furthermore JADE provides a number of additional features such as agent mobility, ontologies and content language support, fault tolerance and web services integration and a rich suite of graphical tools that facilitate the administration of a JADE based application. JADE provides:
\begin{itemize}
 \item An environment where JADE agents are executed.
 \item Class Libraries to create agents using heritage and redefinition of behaviors.
 \item A graphical toolkit to monitoring and managing the platform of Intelligent Agent agents.
\end{itemize}

\subsection*{Apache Ant - TODO}
Apache Ant is a Java library and command-line tool whose mission is to drive processes described in build files as targets and extension points dependent upon each other. The main known usage of Ant is the build of Java applications. Ant supplies a number of built-in tasks allowing to compile, assemble, test and run Java applications. Ant can also be used effectively to build non Java applications, for instance C or C++ applications. More generally, Ant can be used to pilot any type of process which can be described in terms of targets and tasks.

\subsection*{Google Drive - TODO}
Google Drive is a file storage and synchronization service provided by Google, released on April 24, 2012, which enables user cloud storage, file sharing and collaborative editing. Files shared publicly on Google Drive can be searched with web search engines. Google Drive is the home of Google Docs, an office suite of productivity applications, that offer collaborative editing on documents, spreadsheets, presentations, and more.

\subsection*{MongoDB - TODO}
MongoDB is used as the datacommunication layer used to communicate between the MAS and ROS layer. MongoDB is a no-SQL database, which means that it operates with a flexible schema. From the mongo website: "\emph{documents in the same collection do not need to have the same set of fields or structure, and common fields in a collection’s documents may hold different types of data.}"\footnote{http://docs.mongodb.org/manual/core/data-modeling/}

\subsection*{Version Control - TODO}
In this project git was used as version control software. Git is a free and open source distributed version control system designed to handle everything from small to very large projects with speed and efficiency. Git is easy to learn and has a tiny footprint with lightning fast performance. It has features like cheap local branching, convenient staging areas, and multiple workflows.

GitHub\footnote{http://www.github.com/} was used to host the source code for the project. GitHub was founded in 2008 for the purpose of simplifying the sharing of code.

\begin{figure}[h!]
	\caption{GitHub}
	\centering
	\includegraphics[width=10cm]{../images/github-example.jpg}
\end{figure}

\section{Feedback \& Brainstorms}

\subsection{Feedback}
Throughout the research project, feedback was provided by the project supervisor. This was generally supplied in the form of email and through the use of a message board. One exception being a meeting halfway the semester. Whenever a member of the team required input or feedback on something, that member would send an email to the project supervisor or post a message on the message board. The project supervisor would reply to these requests with feedback.

\subsubsection*{Performance review}
Halfway through the research semester, each team member had a performance review with the project supervisor in which the supervisor gave feedback to each member about their personal performance.

\subsection{Brainstorms}
When necessary, brainstorm meetings were held. These brainstorms meetings generally consisted of the project supervisor, the researchers and additionally another small team of interns. Whenever the team and the project supervisor agreed that a brainstorm session was needed a ate would be chosen. In the brainstorm sessions a whiteboard was used to note down everything that was discussed. These sessions generally resulted in a lot of new ideas for the project.

\section{Implementation}

\end{document}

