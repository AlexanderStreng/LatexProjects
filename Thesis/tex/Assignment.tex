\documentclass{../local}
\begin{document}

% ================================== Problem analysis =================================== %
\section{Problem Analysis - TODO}
In this chapter we strive to explain something about the initial situation in current day manufacturing, the solution REXOS offers and try to analyse the problems and challenges with the current REXOS architecture.

\subsection{Initial Situation}
\label{sec:InitialSituation}

\subsubsection{Current manufacturing systems}
%smth about current manufacturing systems and why there so fcked%
Most current day manufacturing systems consist of large production lines able to produce only one type of product at high speeds. A prime example of these are \emph{dedicated manufacturing lines} (DML) which specialize on producing a specific product at a specific capacity. The capacity of the DML and volume of products produced are high, but the diversity of products is low. This results in a high volume - low mix production process. Preparing these kinds of production lines takes time and makes it impossible to quickly develop and test new kinds of products. In order to keep up with current global demand, a more flexible way of producing needs to be invented. That is where FMS\footnote{Flexible manufacturing systems - see definition of terms for more detail.} and RMS\footnote{Reconfigurable Machines} come into play. \textbf{Still needs refs}

\subsubsection{New manufacturing paradigms}
%Lets tell smth about new paradigms, flexible manufacturing, agile etc%
In order to overcome the downsides of static manufacturing, new manufacturing paradigms will have to be invented. A feasible new paradigm is FMS. Y. Koren describes FMS as '\emph{Flexible manufacturing systems (FMS) consist of computer numerically controlled (CNC) machines and other programmable automation and can produce a variety of products on the same system.}\footnote{Koren, Y., 1983, "Computer Control of Manufacturing Systems," McGraw Hill, New York.}\textbf{INSERT PROPER REF SEE FOETNOOT} which results in production machines that can produce a multitude of products. This way, a higher mix of products can be achieved. A second solution is the use of RMs. Y. Koren describes RMS as '\emph{Reconfigurable machines (RMs) are machines whose structures can be changed to provide alternative functionality or/and upgradeable capacity on demand}'\footnote{General RMS Characteristics. Comparison with Dedicated and Flexible Systems}\textbf{INSERT PROPER REF SEE FOETNOOT} which relies on the configurability of the production machines in order to archieve a high mix of products in the production line. A third solution could be the use of 'Grid Manufacturing', which is based on the use of RMS. Grids use a set of reconfigurable machine systems, called ’equiplets’ to offer a diversity of generic services. In this concept an equiplet is a simple, autonomous low cost platform that is highly standardized so it can easily be reconfigured to offer new manufacturing capabilities. Hence, equiplets are able to quickly adapt to new market needs.\footnote{Daniel Telgen et al, Agile Control Architecture for Reconfigurable Manufacturing Systems, Research Centre Technology \& Innovation, HU University of Applied Sciences Utrecht}\textbf{-Probably needs to be a bibitem}


\subsubsection{REXOS as an answer}
%Tell smth about the GOALS and PURPOSES of rexos, and why it was invented.%
The Utrecht University of applied sciences formulated their own solution regarding production. The project is called REXOS for Reconfigurable equiplets operating system, REQs-OS. The heart and soul of the REXOS project is the equiplet, which is a low cost, reconfigurable and autonomous production machine. An equiplet consists of interchangable modules which can perform a specific task. These equiplets will be used in groups to make a complete manufacturing environment. The group of equiplets will have a dynamic logistic set-up, so products and parts can be transported between all systems. A group of equiplets is called a 'grid'.  When operating in a grid, equiplets can cooperate in order to archieve a common goal, producing a product. 

\subsection{Current state of the REXOS architecture}
%smth aboot how the curr architecture was implemented, by ubiquitous students etc%

\subsubsection{Equiplets}
%smth about the current implementation of the equiplets - respresented by a MAS & ROS side etcetc%
\begin{wrapfigure}{r}{0.5\textwidth}
\includegraphics[width=5cm]{../images/Equiplet}
\end{wrapfigure}
Equiplets were first designed by the --\textbf{ DINGEN institute of machinebouw }-- 

When the equiplets first started they had no coherent software whatsoever. Currently the equiplets are all part of the REXOS architecture, and are virtually represented as an entity.


\subsubsection{MAS}
%Tell smth about implementation of the curr MAS system, what purpose it serves, and how its implemented.%
The MAS\footnote{Multi agent system. See abbreviations for more detail.} was designed to be the intelligent side of the the REXOS platform. It incorperates virtual autonomous entities known as agents. An overview of the system is given in image 2
\textbf{--INSERT system overview image here ---}


\subsubsection{ROS}
%Tell smth about implementation of the curr ROS system, what purpose it serves, and how its implemented.%
The current ROS side of the system represents all the hardware in code. All ROS actions are non-autonomous and driven by the intelligent side of REXOS. This means that all MAS-Agents take smart and cognitive decisions and ROS executes these. 
\\
The main focus of this research project was to improve the REXOS platform as it exists today. As described in section \ref{sec:InitialSituation} the architecture was buggy, and not properly implemented. Besides implementing and / or researching a better architecture, a need for new kind of implementation was present as well.

\subsection{Reconfiguring}
%Tell smth about the possibility of reconfiguring, its importance etc%
An important aspect of agile manufacturing is the ability to reconfigure the production machines \textbf{Insert ref here}. The current REXOS architecture didnt provide such a feature. An important focus was to research this, and document it so that future participants of the project would be able to use this information and implement it.


\subsection{Switchable architecture (hierarchical  control vs heterarchical control)}
%Tell smth about the how switchable architectures came to exist, why it can be advantageous%
Another idea that was coined during the project was that of implementing a switchable architecture. This would mean that during normal operation the system operates in a hierarchical fashion, whilst whenever an error occures the system switches to heterarchical control. \textbf{Explain things - Normal heter control $->$ in batches hierarch control.}

\subsection{Simulation}
%smth bout the sim, what we done with it, what its purposes are.%
In order to gather performance data, the decision was made to create a simulator for REXOS. The advantages of running a scenario using a simulation over running a scenario on the actual platform is that a simulated environment is a controlled environment. This means that the amount of products scheduled, the configuration of the grid and any potential errors can be configured. This allows the team to run numerous amounts of variations in grid composition when using the same products. Another aspect that the simulation enables is the testing of the maximum capacity for a grid of an arbitrary size. In turn, this also answers one of the research questions: Is REXOS scalable?

\subsubsection{Test Cases}

\begin{description}
	\item[Exclusive Batch Reservation vs No Reservation] \hfill \\
	In this case, the benefits of reserving equiplets exclusively for certain batch groups will be weighed against the benefits of not making reservations in various situations.
	\item[Heterarchical vs Hierarchical] \hfill \\
	In this case, situations are tested where equiplets fall into an error state and what the benefits are when equiplets switch to heterarchical scheduling and when they do not.
\end{description}


%==================================Research questions===================================%
\section{Research Questions - TODO}
Not all of the research questions that have been formulated in the research plan have been researched and analyzed. Reason for this is that the project leader's needs has changed during the project. Below is a list of the research questions that were originally formulated.
\begin{itemize}
\item How can reconfigurability of equiplets efficiently be implemented into the system?
\item And how will this impact the rest of the system?
	\begin{itemize}
		\item What methods of implementation exist?
		\item What kind of impact do these methods have on the system?
		\item What are the advantages for each method?
	\end{itemize}
\end{itemize}

\begin{itemize}
\item What is the performance and stability like for the system and how can be be improved?
	\begin{itemize}
		\item Where are the bottlenecks for the system?
		\item What is the average time an agent needs to wait for a reply from another agent?
		\item On average, how long does it take to read data from the database?
		\item On average, how long does it take to write data to the database?
	\end{itemize}
\end{itemize}

The research question 'How can reconfigurability efficiently be implemented into the system?' and 'And how will this impact the rest of the system?' has not been researched. Reason for this is because in brainstorm sessions conclusions have been made that other issues had to be looked in to before above questions could be answered. 

When an equiplets needs reconfiguration, this will impact products that needs producing, especially schedules of these products will be affected. There was no scheduling alogrithm implemented in REXOS before the project. The project leader's decision was that a scheduling algorigthm in REXOS was more important than reconfiguration of an equiplet. A scheduling algorithm for REXOS was already researched and described in \textbf{REFERENCE TO PAPER 70}. This resulted in the following new research questions:

\begin{itemize}
\item Is the scheduling algorithm defined in \cite{} suitable for implementation in REXOS?
	\begin{itemize}
	\item Which parts of the algorithm is suitable for REXOS and what parts are not?
	\item If some parts can't be added directly, what is a better way to implement these?
	\end{itemize}
\end{itemize}

%==================================Goals===================================%
% not 100% sure this section should be here, half and half
\section{Goals - TODO}
In short the goals of this project are:
\begin{itemize}
\item gaining a stable platform to have reliable research results and that future projects can continue without big issues
\item Getting a picture of how the current platform performs and if the platform needs improvement or change
\item Create a proposition on how to implement reconfigurability within the REXOS platform, so one of the key features of a future REXOS can be developed.
\end{itemize}

\subsection{Stable REXOS platform}
Prior to the research that has to be done. The current platform's stability has to be improved. This has to be done because testing solutions or performing tests will be more accurate and reliable. Also when a stable platform is provided for the next project, developing or researching using that platform will be easier and more reliable. Stabilizing a platform can be done continuously, but in this project it will have a time limit and when the project supervisor thinks that the platform is stable enough.

\subsection{Performance}
After the platform has been stabilized, the platform will be tested. The goal of testing the platform is to get a picture on how 'good' the current platform is. A conclusion can be made out of the performance results if certain parts of the REXOS architecture has to be changed or improved. An important feature that has to be reached of REXOS is scalability. Untill today there has not been any research to the scalability of REXOS. With the performance results, the scalability of the current REXOS platform can be described.

\subsection{Reconfigurability}
Reconfiguring an equiplet is one of the main features of a future REXOS platform in a factory. Up until this moment there is not yet any research of implementing reconfigurability of an equiplet within the REXOS platform. With the research to reconfigurability within the REXOS platform, the goal is to create a proposition of how reconfigurability can be implemented on the MAS level of REXOS. The document will describe what has to be changed within the REXOS architecture in order to make reconfigurability work on a higher level.

\end{document}
