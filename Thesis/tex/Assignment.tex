\documentclass{../local}
\begin{document}

% ================================== Problem analysis =================================== %
\section{Problem Analysis}
In this chapter we strive to explain something about the initial situation in current day manufacturing, the solution REXOS offers and try to analyse the problems and challenges with the current REXOS architecture.

\subsection{Initial Situation in manufacturing}
\label{sec:InitialSituation}

\subsubsection{Current manufacturing systems - TODO}
%smth about current manufacturing systems and why there so fcked%
Most current day manufacturing systems consist of large production lines able to produce only one type of product at high speeds. A prime example of these are \emph{dedicated manufacturing lines} (DML) which specialize on producing a specific product at a specific capacity. The capacity of the DML and volume of products produced are high, but the diversity of products is low. This results in a high volume - low mix production process. Preparing these kinds of production lines takes time and makes it impossible to quickly develop and test new kinds of products. In order to keep up with current global demand, a more flexible way of producing needs to be invented. That is where FMS\footnote{Flexible manufacturing systems - see definition of terms for more detail.} and RMS\footnote{Reconfigurable Machines} come into play. \textbf{Still needs refs}

\subsubsection{New manufacturing paradigms - TODO}
%Lets tell smth about new paradigms, flexible manufacturing, agile etc%
In order to overcome the downsides of static manufacturing, new manufacturing paradigms will have to be invented. A feasible new paradigm is FMS. Y. Koren describes FMS as '\emph{Flexible manufacturing systems (FMS) consist of computer numerically controlled (CNC) machines and other programmable automation and can produce a variety of products on the same system.}\cite{Koren} which results in production machines that can produce a multitude of products. This way, a higher mix of products can be achieved. A second solution is the use of RMs. Y. Koren describes RMS as '\emph{Reconfigurable machines (RMs) are machines whose structures can be changed to provide alternative functionality or/and upgradeable capacity on demand}'\footnote{General RMS Characteristics. Comparison with Dedicated and Flexible Systems}\textbf{INSERT PROPER REF SEE FOETNOOT} which relies on the configurability of the production machines in order to archieve a high mix of products in the production line. A third solution could be the use of 'Grid Manufacturing', which is based on the use of RMS. Grids use a set of reconfigurable machine systems, called ’equiplets’ to offer a diversity of generic services. In this concept an equiplet is a simple, autonomous low cost platform that is highly standardized so it can easily be reconfigured to offer new manufacturing capabilities. Hence, equiplets are able to quickly adapt to new market needs.\cite{GRIDPARADIGM}


\subsubsection{REXOS as an answer}
%Tell smth about the GOALS and PURPOSES of rexos, and why it was invented.%
The Utrecht University of applied sciences formulated their own solution regarding production. The project is called REXOS for Reconfigurable equiplets operating system, REQs-OS. The heart and soul of the REXOS project is the equiplet, which is a low cost, reconfigurable and autonomous production machine. An equiplet consists of interchangable modules which can perform a specific task. These equiplets will be used in groups to make a complete manufacturing environment. The group of equiplets will have a dynamic logistic set-up, so products and parts can be transported between all systems. A group of equiplets is called a 'grid'.  When operating in a grid, equiplets can cooperate in order to archieve a common goal, producing a product. 

Having these low cost and agile production robots, a large variety of products can be produced. This is also the main vision of REXOS. Having a manufacturing environment that can produce products of low amounts and with a high variety between them ( low volume, high mix ).

\subsubsection{Equiplets}
%smth about the current implementation of the equiplets - respresented by a MAS & ROS side etcetc%
\begin{wrapfigure}{r}{0.5\textwidth}
\includegraphics[width=5cm]{../images/Equiplet}
\end{wrapfigure}
Equiplets were first designed by the Institute for Engineering \& Design at the University Utrecht.

The first versions of the equiplets were standalone robots that only had software to move the module that was installed on. The REXOS platform did not exist at that time. Now with the REXOS platform, equiplets are a lot more sophisticated.

As stated in the previous section, equiplets need to be flexible and reconfigurable, but also applicable for industrial systems. In order to achieve this, a hybrid system of MAS and ROS is used\cite{MASTA2013}.

The current version of the equiplet now is represented by an agent entity, which is responsible for the negotiation between products and interpretation of tasks, and a ROS side, which is responsible for the hardware. The agent entity will respond to negotiations from product agents. These negotiations involve scheduling of steps ( product steps ) that the product agent needs done by the equiplet. When these negotiations are done, the agent entity of the equiplet sends data to the ROS part to execute the product step.

Using this software for the equiplet, it is now capable to perform simple tasks with a high precision and high flexibility. 

\subsection{GRID}
\begin{figure}[h!]
	\centering
	\includegraphics[width=12cm]{../images/Grid.png}
	\caption{A equiplet grid}
\end{figure}


\subsection{Reconfiguring - TODO}
%Tell smth about the possibility of reconfiguring, its importance etc%
An important aspect of agile manufacturing is the ability to reconfigure the production machines \textbf{Insert ref here}. The current REXOS architecture didnt provide such a feature. An important focus was to research this, and document it so that future participants of the project would be able to use this information and implement it.


\subsection{Switchable architecture (hierarchical  control vs heterarchical control) - TODO}
%Tell smth about the how switchable architectures came to exist, why it can be advantageous%
Another idea that was coined during the project was that of implementing a switchable architecture. This would mean that during normal operation the system operates in a hierarchical fashion, whilst whenever an error occures the system switches to heterarchical control. \textbf{Explain things - Normal heter control $->$ in batches hierarch control.}

\subsection{Simulation}
%smth bout the sim, what we done with it, what its purposes are.%
In order to gather performance data, the decision was made to create a simulator for REXOS. The advantages of running a scenario using a simulation over running a scenario on the actual platform is that a simulated environment is a controlled environment. This means that the amount of products scheduled, the configuration of the grid and any potential errors can be configured. This allows the team to run numerous amounts of variations in grid composition when using the same products. Another aspect that the simulation enables is the testing of the maximum capacity for a grid of an arbitrary size. In turn, this also answers one of the research questions: Is REXOS scalable?

\subsubsection{Test Cases}

\begin{description}
	\item[Exclusive Batch Reservation vs No Reservation] \hfill \\
	In this case, the benefits of reserving equiplets exclusively for certain batch groups will be weighed against the benefits of not making reservations in various situations.
	\item[Heterarchical vs Hierarchical] \hfill \\
	In this case, situations are tested where equiplets fall into an error state and what the benefits are when equiplets switch to heterarchical scheduling and when they do not.
\end{description}


%==================================Research questions===================================%
\section{Research Questions}
Not all of the research questions that have been formulated in the research plan have been researched and analyzed. Reason for this is that the project supervisor's needs have changed during the project. Below is a list of the research questions that were originally formulated.
\begin{itemize}
\item How can reconfigurability of equiplets efficiently be implemented into the system?
\item And how will this impact the rest of the system?
	\begin{itemize}
		\item What methods of implementation exist?
		\item What kind of impact do these methods have on the system?
		\item What are the advantages for each method?
	\end{itemize}
\end{itemize}

\begin{itemize}
\item What is the performance and stability of the system and how can this be improved?
	\begin{itemize}
		\item Where are the bottlenecks for the system?
		\item What is the average time an agent needs to wait for a reply from another agent?
		\item On average, how long does it take to read data from the database?
		\item On average, how long does it take to write data to the database?
	\end{itemize}
\end{itemize}

The research question 'How can reconfigurability efficiently be implemented into the system?' and 'And how will this impact the rest of the system?' has not been researched. The reason for this is that conclusions have been made during brainstorm sessions that other issues had to be taken care of before above questions could be answered. 

When an equiplet needs reconfiguration, this will impact products that need producing, especially the schedules of these products will be affected. There was no scheduling alogrithm implemented in REXOS before the project. The project supervisor's decision was that a scheduling algorigthm in REXOS was more important than reconfiguration of an equiplet. A scheduling algorithm for REXOS was already researched and described in the paper: 'Multiagent-based agile manufacturing\cite{PAPER70}'. This resulted in the following new research questions:

\begin{itemize}
\item Is the scheduling algorithm defined in paper: 'Multiagent-based agile manufacturing\cite{PAPER70}' suitable for implementation in REXOS?
	\begin{itemize}
	\item Which parts of the algorithm are suitable for REXOS?
	\item If some parts can't be implemented directly, what is a better way to implement these?
	\item What is the performance of this scheduling algorithm?
	\end{itemize}
\end{itemize}

Another research question came up during the project that had a high priority. During brainstorm sessions questions came up concerning whether REXOS should have a hierarchical or heterarchical setup. Stated was that both setups had their own advantages and disadvantages. Choosing a right setup is neccesary. 

\begin{itemize}
\item Is REXOS best suited for a hierarchical or a heterarchical setup?
	\begin{itemize}
	\item Is it possible to combine these setups, and what can be the advantages of doing so?
	\end{itemize}
\end{itemize}

The addition of this new research question resulted in a lower priority for the 'performance and stability' research question.
%==================================Goals===================================%
\section{Goals}
In short the goals of the project are:
\begin{itemize}
\item having a stable platform to have reliable research results and that future projects can continue without big issues, also the performance has been improved
\item having a new implemented scheduling algorithm
\item answering the question if a hierarchical or heterarchical setup is better for REXOS. 
\end{itemize}
Below these goals are described more in detail.

\subsection{Stable REXOS platform}
Prior to the research that has to be done. The current platform's stability has to be improved. This has to be done because testing solutions or performing tests will be more accurate and reliable. Also when a stable platform is provided for the next project, developing or researching using that platform will be easier and more reliable. Stabilizing a platform can be done continuously, but in this project it will have a time limit and when the project supervisor thinks that the platform is stable enough.

After the platform has been stabilized, the platform will be tested. After testing, there will be a picture of where the bottlenecks of the system are. The goal is that these bottlenecks will be resolved.

\subsection{Scheduling}
Current implementation of scheduling for products at equiplets don't exist yet in REXOS.  The goal is that at the end of the project an efficient scheduling algorithm will be functional and working in REXOS. This implementation can then be tested afterwards for efficiency with proper tools.

\subsection{Hierarchical or heterarchical}
Choosing the right setup for REXOS could determine the flexibility, scalability and efficiency of the platform. Using testing tools with different cases, a decision can be made what setup is better for certain cases. The goal of this project is to find out when hierarchical is better than heterarchical and vice versa.

\end{document}
