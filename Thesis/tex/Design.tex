\documentclass{../local}
\begin{document}

In this chapter we describe the process of designing and the resulting design of certain elements from the current implementation of REXOS that are subject to change. This includes parts of the architecture that need replacing, such as the current implementation of planning \& scheduling.

\section{REXOS Architecture}
Throughout the whole project the main subject was constantly improving and modifying the current REXOS architecture. During the first part of the research a deadline was set to write and research feasable demo's and achieve a stable architecture for the precision fair. After the precision fair, a simulation had to be written in order to research the advantages of reserving batches (hierarchical control).

The REXOS architecture as it was finished last semester was not stable enough to perform without failure. First of all there wasnt an implementation for other modules than the current pen-module and most of the work was poorly documented. The current implementation also lacked proper scheduling. 

\subsection{Agent communication}
\label{sec:AgentComm}
On the MAS side of the system, certain things needed to change. One of the biggest problems was getting a proper insight in to how the system functioned and how communication was implemented. After some research conclusions figure \ref{AgentCommunicationOverview} was drawn.
\begin{figure}[h!]
	\includegraphics[width=1.1\linewidth]{../images/AgentCommunicationOverview.pdf}
	%we might want to increase the text on this image..%
	\caption{Agent communication overview}
\label{fig:AgentCommunicationOverview}
\end{figure}
Marked in red are the changes that need to be made to make all communication FIFA \textbf{insert footnote / ref} compliant.

\subsection{Knowledge database}

\subsection{Planning}
The planning algorithm used in REXOS is taken from \emph{L. van Moergestel’s paper ‘Multiagent-based agile manufacturing: from user requirements to product.} \textbf{- INSERT QUOTE HERE - } and the batch scheduling algorithm is described below. Scheduling is done by the product agent, which represents a product.
The product consists of steps, which are specific tasks defined by a set of parameters. Think of screwing, glueing or welding. All the equiplets in a grid can perform a certain set of these tasks. The ability to perform such a task is called a capability.  Each equiplet only has a limited set of these capabilities.

In the simulation a grid consisting of equiplets is defined. Each equiplet has its own set of capabilities. Now consider a product with an n amount of product steps($\phi$):\\

$< \phi_1, \phi_2, ... \phi_n >$\\

Once the product (agent) spawns, it will match all possible equiplets to complete all of its steps. Meaning that it compiles a collection containing equiplets combined with the product steps they can perform. That seems like a lot, but the equiplets do not have a large set of capabilities. The resulting collection is:\\

$< E1(\phi_1, \phi_3), E2(\phi_2, \phi_3, \phi_1), E3(\phi_1) >$\\

Once this collection has been compiled,  the product(agent) starts negotiating with the equiplets. It informs with each of the equiplets to evaluate whether or not the step can be performed with the given parameters at the given equiplet.  Once all equiplets have been queried and matched, the actual planning begins.
The first step in the planning consist of reducing the transition time between product steps. First a production matrix is constructed. The rows represent the equiplets while the columns represent the product steps.
goog
\begin{table}[htbp]
\centering
\normalsize
\begin{tabular}{|l|l|l|l|l|l|}
\hline
	& $\phi$1 & $\phi$2 & $\phi$3 & $\phi$4 & $\phi$5 \\
\hline
Equiplet 1 & 1.0 & 0.0 & 1.0 & 1.0 & 1.0 \\
\hline
Equiplet 2 & 0.0 & 1.0 & 0.0 & 0.0 & 0.0 \\
\hline
Equiplet 3 & 0.0 & 0.0 & 0.0 & 0.0 & 1.0 \\
\hline
Equiplet 4 & 1.0 & 0.0 & 1.0 & 1.0 & 0.0 \\
\hline
\end{tabular}
\caption{Initial production matrix}
\end{table}

First off, all equiplets that are capable of performing a step have their value raised to 1.
The next step is to minimize transition. 


To prevent excess transition between equiplets during manufacturing, all equiplets with sequential steps being performed on the same equiplet have their value raised by the length of the sequence -1.  Resulting in the following matrix:

\begin{table}[htbp]
\centering
\normalsize
\begin{tabular}{|l|l|l|l|l|l|}
\hline
	& $\phi$1 & $\phi$2 & $\phi$3 & $\phi$4 & $\phi$5 \\
\hline
Equiplet 1 & 1.0 & 0.0 & 3.0 & 3.0 & 3.0 \\
\hline
Equiplet 2 & 0.0 & 1.0 & 0.0 & 0.0 & 0.0 \\
\hline
Equiplet 3 & 0.0 & 0.0 & 0.0 & 0.0 & 1.0 \\
\hline
Equiplet 4 & 1.0 & 0.0 & 2.0 & 2.0 & 0.0 \\
\hline
\end{tabular}
\caption{Production matrix after transition optimisation}
\end{table}

As is shown in figure (above) E1 is able to perform step 3, 4 \& 5. The corresponding values where increased by 2 ( sequence length is 3, minus 1 ).

Another easy but important optimization is load balancing. All equiplets are responsible for their own schedule which makes calculating the load at any given time feasible.
Consider a product querying a equiplet to calculate its load for scheduling its 57’th step. This step has to be carried out not at release time, but at release time + $\Delta$ ( where $\Delta$ delta is the time the previous 56 steps will take, including travel time. ) meaning that the equiplet has to calculate load over that time ( called a window).

Consider the following 2 schedules:

\begin{figure}[h!]
\flushleft
	\includegraphics[width=1.1\linewidth]{../images/EquipletLoadWindowExample.jpg}
	%we might want to increase the text on this image..%
	\caption{Equiplet schedules}
\end{figure}

If the equiplet would calculate load over release time + window length, it would not be a correct calculation. In above image equiplet 1 has a load of 90\% whereas equiplet 2 has a load of 20\%.
In order to favor the equiplets with least load, all numbers in the matrix are multiplied with ( 1 – (load) ).

Applying this to our current production matrix, and a imaginary constant load ( E1 – 60\%, E2 – 20\%, E3 – 40\% and E4 – 10\% ), the matrix will result in:

\begin{table}[htbp]
\centering
\normalsize
\begin{tabular}{|l|l|l|l|l|l|}
\hline
	& $\phi$1 & $\phi$2 & $\phi$3 & $\phi$4 & $\phi$5 \\
\hline
Equiplet 1 & 0.4 & 0.0 & 1.2 & 1.2 & 1.2 \\
\hline
Equiplet 2 & 0.0 & 0.8 & 0.0 & 0.0 & 0.0 \\
\hline
Equiplet 3 & 0.0 & 0.0 & 0.0 & 0.0 & 0.6 \\
\hline
Equiplet 4 & 0.4 & 0.0 & 1.2 & 1.2 & 0.0 \\
\hline
\end{tabular}
\caption{Production matrix after load balancing}
\end{table}

(steps Ø 3,Ø 4 \& Ø 5 on E1 are multiplied by ( 1 – 0.6 ) which results in 3 * 0.4  )
As seen in the figure above, E1 which has a load of 60\% is suddenly not the best option for steps 3 \& anymore. This way, a proper load balance between the equiplets is achieved.

Another important optimization is reducing the travel time between steps. Much like reducing the transition time between steps that can be executed on the same equiplet, this optimization is meant to reduce travel time within the grid. In order to do so, a distance matrix is utilized. This matrix has equiplets in both the rows as columns, and travel times between these as values ( in amount of timeslots required to travel  from 1 to another ):

\begin{table}[htbp]
\centering
\normalsize
\begin{tabular}{|l|l|l|l|l|}
\hline
	& Equiplet 1 & Equiplet 2 & Equiplet 3 & Equiplet 4  \\
\hline
Equiplet 1 & x & 3.0 & 1.0 & 2.0 \\
\hline
Equiplet 2 & 3.0 & x & 6.0 & 3.0 \\
\hline
Equiplet 3 & 1.0 & 6.0 & x & 6.0 \\
\hline
Equiplet 4 & 2.0 & 3.0 & 6.0 & x \\
\hline
\end{tabular}
\caption{Grid distance matrix}
\end{table}

Now, utilising the production matrix, the product will try and use a path of least resistance algorithm to find the most suitable path within the grid.\\

Ignoring all 0 values,  the following possible paths are presented:\\

Combined with the weight of the values in the production matrix, a couple of feasible paths are calculated. These parts are then scheduled sequentially. ( taking the best path first ).

\subsection{Scheduling}
\label{sec:Scheduling}
After the production plan has been made, the product agent will then schedule its plan at the corresponding equiplets. The schedule of an equiplet has to be easily accessible and easily to manipulate. Also, calculating the load of an equiplet or gaps in the schedule needs to be fast.

\subsubsection{Representation of an equiplet's schedule }
In \textbf{In paper70 sections 2.1 and 5.6 } a planning algorithm is defined. This algorithm has a major drawback however, it is not suitable for use in the current system. 

For REXOS we took another approach to avoid this issue. We abandoned the idea of using a circular buffer, but we still need to store the schedules of equiplets. Parts that have to be saved on these schedules are steps of a product and time information about when that step will be executed and how long that will take. 

In reality an equiplet lives from product step to product step. When an equiplet does not have a product step planned, it doesnt have to do anything special. Translating this to software means that having a linked list of product steps is enough to let the equiplet function. Using a circular buffer will store every timeslot that will happen, meaning that these stored timeslots could be empty. Having a load of 20\% would mean that that 80\% of the timeslots would be empty. This is lost memory and utilizing a linked list would eliminate this performance problem.

Calculating available free time slots for a given product step would be faster using the aforementioned solution. Rather than iterating through and counting all of the elements when using a circular buffer, a free time slot can be calculated between 2 elements of the linked list solution. The same applies to calculating the load of an equiplet.

\subsubsection{Locking a schedule of an equiplet}
\label{sec:schedulelock}
In REXOS, there is a high possibility that multiple product agents will be scheduling at the same equiplet simultaneously. When this happens, it is likely that the product agents will try to schedule at the same time. Resulting in concurrency errors.

To prevent these situations, the planning and scheduling operations have to be atomic actions \textbf{-reference from L. Moergestel paper-}. Once a product agent starts requesting information of the schedule of an equiplet with the intention of scheduling new product steps, the equiplet will lock its schedule and give the key to that product agent. Other product agents needing to schedule cannot request any information of the schedule of this 'locked' equiplet's schedule. Once the product agent that received the key of the equiplet's schedule is done scheduling, the product agent will unlock this schedule and other product agents will be able to schedule on this equiplet again.

Using the above mentioned planning algorithm will result in a lot of schedules used and locked by the product agent, but not all of these schedules will be chosen to produce on. Given the following situation: There are three equiplets providing capability one ( drilling ). When a product agent spawns and wants to schedule, whose product steps needs the capability drilling, it will lock all of the three equiplet's schedule. When a second product agent wants to schedule that needs the same capability, it cannot lock any of the required equiplets.

The most efficient way of resolving this issue is to let the second product agent wait until the equiplet's schedules are unlocked. The main reason for this is that the planning algorithm has the best results when all of the possible equiplets for the required product steps are used. Excluding some of the equiplets from the algorithm will eventually result in an inefficient production route for some products.

Using the above solution for the problem could cause a major software deadlock in the MAS system. A situation can occur that multiple product agents need to wait for each other. The following figure will illustrate this:

\begin{center}
\includegraphics{../images/Scheduledeadlock}
\end{center}

In the above picture two product agents are spawned at the same time and need the same equiplets. In the MAS system, the product agent will send lock request messages to the equiplets. In the Jade framework, it is possible due to latency that the last sent lock request message ( the message to EQ$_3$ from PA$_2$) will be received before the lock request message from PA$_1$. In above situation PA$_1$'s messages are received in order from EQ$_1$ to EQ$_3$ and PA$_2$'s messages from EQ$_3$ to EQ$_1$. For this example messages are processed at 1 message per tick.

After the first tick, PA$_1$ has the lock of EQ$_1$ and PA$_2$ has the lock of EQ$_3$. After the second tick a race condition appears and in this case PA$_1$ gets the lock of EQ$_2$, PA$_2$ will not get the lock of EQ$_2$ and will need to wait for it at the end, but PA$_2$ will continue since he needs EQ$_3$. Now for the next tick, there is a problem. PA$_1$ requests the lock of EQ$_3$, but is not given the lock because PA$_2$ already has the lock. For the same reason the lock of EQ$_1$ cannot be given to PA$_2$. Given the above solution we end up in a deadlock because PA$_1$ will wait for the lock of EQ$_3$ to release and PA$_2$ will wait for the release of the lock of EQ$_1$.

To solve this problem, this situation has to be detected and resolved. Detection can be done as following: When an product agent (PA$_1$) stumbles upon a locked equiplet (EQ$_3$), it will request the address of the product agent that locked EQ$_3$, resulting in the address of PA$_2$. PA$_1$ will then send a message to PA$_2$, notifying PA$_2$ that PA$_1$ is in need of the locked EQ$_3$. When PA$_2$  will detect a locked EQ$_1$ that is locked by PA$_1$, by asking EQ$_1$ the address of the locker, PA$_2$ will compare the address returned with its notification list and will detect the deadlock of two product agents that need each others locked equiplets.

There is also another case where the above solution can probably not detect a deadlock with three or more product agents. Given the following case with three product agents: PA$_1$ needs EQ$_1$ and $_2$, PA$_2$ needs EQ$_2$ and EQ$_3$ and PA$_3$ needs EQ $_3$ and EQ$_1$. all three product agents will lock their first equiplet at the same time. At the the second lock round, A circle of waits will be created ( see figure 2 ), but two product agents aren't waiting on eachother like the first case. 

\begin{center}
\includegraphics{../images/CircleDeadlock}
\end{center}

Solving this problem can be done by forwarding waiting notifications. Given the situation is that PA$_1$ already sent a waiting notification to PA$_2$. When PA$_2$ needs to send the waiting notification to PA$_3$, it will pass its waiting notification queue along to PA$_3$. PA$_3$ will do this also to PA$_1$. When PA$_1$ processes the incoming message with the forwarded waiting notification queue, it will notice that his own product agent name is in the list. Meaning that the waiting notification has made a circle, resulting in a deadlock.

Resolving the deadlock is the same for both cases, the detector will start a negotiation with all of the involved product agents. A choice will have to be made who will start with producing. Based on the needs of the current grid setup, the information needed for making the choice can vary. For example: if deadlines are a high priority, the product agent with the earliest deadline will be chosen first, or if the product with the shortest produce time are useful to let them schedule first. Further research on this topic can be done to provide an answer. The chosen equiplet will be given a queue of the remaining product agents sorted on who goes when and will be passed on until all of the 'deadlocked' product agents have finished scheduling.

\subsection{Equiplet Error}
In real life situations, the hardware or software of equiplets can fail at any given time. There are two kinds of these so called errors, hardware failures and software failures. Software errors can occur when runtime undefined exceptions could occur. When the software cannot handle this exception, it will set the equiplet in error. Hardware errors will occur when the hardware does not act the way it is asked by the software. When the software detects odd behaviour of the hardware and cannot resolve it, the software will also set the equiplet in error.

When one of these errors occur, depending of the state of the equiplet, a number of decisions has to be made. The equiplet has to made decisions about the following: 
\begin{itemize}
\item When products have planned their steps at the failing equiplet, what happens to their planned steps?
\item When the equiplet is producing, what does he have to do with the current product?
\end{itemize}

the equiplet will cancel its current action and notify the products that have been scheduled at this equiplet, that the equiplet has failed. With the current implementation, the products affected will reschedule all of its steps. In real life, it is possible that the error will be resolved early and some products will not be affected since they are scheduled later on. But the equiplet does not know when the error will be resolved. Later implementations can improve this by letting a schedule get out of its deadline ( soft deadlines ) or predicting when the error will be resolved by learning from previous errors.

The difference between a software error and a hardware error lies within the ability to solve these errors autonomous. In most hardware error cases, some hardware needs replacing ( often done by an operator), and the equiplet can’t do this himself. Software errors can always be resolved by an equiplet, whether it will be processed and execution will be adapted or by a reset of the software. Both of these errors can result in the destruction of the product it was working on, i.e. the time lost due to error handling can cause the product to fail when it needs to glue 2 parts together and the glue will be dried before the equiplet could restore. It always depends on the situation if the product will be damaged during a software or hardware error.

When an error happens, the equiplet can only take care of itself, but not for the product. The product agent itself needs to handle according to the situation. I.e. when the equiplet has damaged the product it was working on, the product agent has to restart its product, or just report it has failed. When a product was scheduled at the failing equiplet, it needs to reschedule its steps.

\section{Simulation}
Collecting performance data on the REXOS platform was neccessary, however, running sets and sets of test cases on the hardware would be very time consuming. For this reason, the decision was made to develop a simulator for the platform. This simulator would be capable of running test cases without requiring the use of the hardware.

\subsection{Mechanics}
The simulator that was developed uses a concept typically found in games, a single threaded update loop in which the simulation is updated in each iteration of that loop. A list of updateables is tracked and these are updated during each iteration, or tick, of the simulation.

The simulator has a graphical user interface which runs on it's own thread, this to preserve the purity of the simulation, and also to be able to keep the user interface updated with live feedback during the simulation.

\subsubsection*{Scheduling}
The scheduling algorithm implemented into the simulation was developed during the research semester for the REXOS platform. The algorithm has not actually been implemented into the platform yet though, and so this opportunity was used as a test for the scheduling algorithm as well. More information on the algorithm itself can be found in section ~\ref{sec:Scheduling}.

\subsubsection*{Data Acquisition}
Data is gathered using data collectors. These collectors have a collectData method, which is specific for the type of data being collected. For example, there is a data collector for equiplets and one for products. When told to update, the collector checks whether it is time to gather new data, in the case that it is, it tries to process it's specific type of updateables. In the case of equiplets, it retrieves a grid object, and using that object, retrieves all the equiplets in the grid, after which it queries each individual equiplet for it's statistics. In the case of the product data collector, the list of updateables is filtered to get all the product objects, which the collector then uses to gather data from. For an equiplet, the load, current load, products spawned, products simultaneously in grid, and products failed (total and batch) metrics are recorded and collected. The load is a metric which is calculated by looking at the future schedule window and determining which percentage of that window is scheduled. The current load however, looks at the past interval time between data collections, and it checks whether the equiplet has actually been producing during that time, it then calculates the percentage that he equiplet was busy for. Products spawned indicates the total number of products spawned at the time of collection. The products simultaneously in grid is as the name suggests, the number of products active in the grid at the time of data collection. The products failed metric records the amount of products that fail during the simulation, and of this number, how many are batch products.

\subsubsection*{Spawning Products}
Product spawner, batch spawner and dynamic product spawner are the different types of spawners implemented for the simulation. The product spawner and batch spawner read their respective files for product and batch definitions before then adding those products and batches to the simulation process. The product and batches have intervals defined in the files (\emph{for more information on the file format, please check section ~\ref{sec:di-gui}.}), this interval is used as a timer to see when the next wave of products or batches should be spawned.

The dynamic product spawner operates in a different manner to the others. It uses the parameters target load, min/max product length and chance of spawning. What it does is continually spawn products in order to reach the specified target load, the products spawned vary in number of product steps, which is what the min and max product length parameters are for. Finally, the chance of spawning parameter is used for a chance calculation, to determine whether or not a particular product spawns or not. There is also an option to simulate peaks in the grid.

\subsection{GUI - TODO}
\subsubsection{Requirements}
These requirements were set up using the MoSCoW method.

\paragraph{Must Have}

\begin{enumerate}
\item Simulation Interface \hfill \\
The interface used to control the simulation. This interface will also be used to display any feedback given.

\item Start simulation \hfill \\
A way to start running the simulation.

\item Insert error into equiplet \hfill \\
Used to put a virtual equiplet into a state of error. This can be used to simulate hardware or software failure.
\end{enumerate}

\paragraph{Should Have}
\begin{enumerate}
\item Stop simulation \hfill \\
A way to stop running the simulation.

\item Read configuration files \hfill \\
When running a large number of test simulations, it would save time if the grid configuration, products and capabilities could be read from a file, rather than needing to re-define the composition every time a new simulation is started.

\item Open simulation editor \hfill \\
While not strictly necessary, having a button to open the configuration editor saves time and unnecessary complexity when trying to run the application.

\item Simulation Editor Interface \hfill \\
This interface will be used to quickly create or modify a configuration. It has the ability to create virtual objects.

\item Create Objects \hfill \\
The sole function of the editor interface is to create virtual objects, therefore it needs to have that feature.

\item Assign newly made objects to one currently being created \hfill \\
Given the following example:

\begin{enumerate}
\item Create a capability called 'Saw'
\item Create a virtual equiplet, which has the 'Saw' capability
\end{enumerate}

To create a new virtual equiplet, the interface must have a way of selecting previously made capabilities. This also counts for products, batches and the grid.

\item Output files \hfill \\
The editor exists to simplify the creation of configuration files, making the feature to output the configuration files a must have.

\item Read and parse files \hfill \\
While it is possible to create a new configuration rapidly using the interface, if one has already been made before, it would be easier to read that configuration and make an adjustment rather than re-creating it from scratch.
\end{enumerate}

\paragraph{Could Have}
\begin{enumerate}
\item Visualize grid \hfill \\
When simulating a virtual grid of equiplets, it would be more intuitive to see the grid visualized instead of manually checking a single equiplets' position within the grid. It is however, not required for the application to perform.

\end{enumerate}

\subsubsection{Visual Design}

\begin{center}
	\includegraphics[width=12cm]{../images/EditorGUI-MainScreen}
	\captionof{figure}{The Simulation Editor Interface}
\end{center}

\end{document}
