\documentclass{../local}
\begin{document}

\section{Project}
\subsection{Background}
In 2008 a project called HUniversal Production was started by E. Puik. This project's goal was researching agile manufacturing\cite{AGILEMAN} and developing a working platform based on this manufacturing paradigm. Since then, the research into agile manufacturing has become larger and a lot more research is needed on different parts of the whole platform.
 
The first prototype of the platform consisted of a single prototype 'Equiplet' named the "HUniplacer". While developing this first prototype the focus was put on the modular design of the hardware.
 
In 2012, the idea of reconfigurable equiplets was introduced. Being able to safely remove, replace or add new modules to a running system. This led to the development of MAST, or MAchine STates. Using MAST simplifies checking whether a module is busy or idle, but more importantly, it allows modules to be put in a safe state, making removal possible and safe.

\subsection{Team}

\begin{table}[htbp]

\centering
\normalsize
\begin{tabular}{|l|l|}
\hline
Name & Role\\
\hline
E. Puik & Founding father\\
\hline
D. Telgen & Project supervisor\\
\hline
A. Streng & Team supervisor, researcher/software developer and scrum master\\
\hline
D. Jenkins & researcher/software developer\\
\hline
R. Scheefhals & researcher/software developer\\
\hline
A. Hustinx & intern/software developer\\
\hline
T. Bakker & intern/software developer\\
\hline
G. Hakopian & intern/software developer\\
\hline
\end{tabular}
\caption{Project team}
\end{table}
The researchers involved in the project are students at the university 
in Utrecht. They work directly under D. Telgen in a researching capacity.

\section{Organization}
\subsection{Knowledge Center for Technology and Innovation}
The research center is a place for where innovation can take place for the Utrecht region. Additionally, the center works together with the University of Applied Sciences Utrecht doing research for practical solutions.

The research center uses the expertise available within higher education to undertake projects and find solution at low costs. Complex issues can be resolved by collaborating with other knowledge institutes.

Research results are disseminated widely by mutual agreement in order to contribute to the innovation capacity for the region. Students from the bachelor studies are aided by lecturers and guided by professors while conducting projects and research.

\subsection{Lectureship Microsystemtechnology/Embedded Systems}
Microsystems, such as autonomous sensor systems are difficult to produce, as very high precision is required during manufacturing. This causes the production of these systems to be slow compared to the average product. The lectureship is involved in the optimization of product designs and design procedures of these systems. Another area the lectureship is involved in is the modularization of production processes.

An example of the research done by the lectureship is a large project by the name of 'Submissive Product Design'. Under this project ten seperate research projects were concluded, which concerned the development of product designs and procedures.

The lectureship focuses on product improvement and the industrialisation of microsystems. It is located within the knowledge center for technology and innovation located at Oudenoord 700 Utrecht.
% explain more about the lectureship, what it does, rather than what the definition of it is
\end{document}