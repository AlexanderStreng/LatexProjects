\documentclass{../local}
\begin{document}

\section{Project}
\subsection{Background}
In 2008 a project called HUniversal Production was started by E. Puik. This project was started to research agile manufacturing\cite{AGILEMAN} and to develop a working platform based on this manufacturing paradigm. Since then, the research into agile manufacturing has become larger and a lot more research was needed on different parts of the whole platform.
 
The first prototype of the platform consisted of a single machine capable of picking up and placing different kind of objects. While developing this first prototype a lot of focus was on the architecture of the platform. The architecture was based on a hybrid architecture consisting of ROS and JADE.
 
In 2012, the idea of reconfigurable equiplets was introduced. Being able to safely remove, replace or add new modules to a running system. This led to the development of MAST, or MAchine STates. Using MAST simplifies checking whether a module is busy or idle, but more importantly, it allows modules to be put in a safe state, making removal possible.
 
Currently, the platform uses MAS and ROS, which communicate through blackboards to create an agile manufacturing system. Reconfigurability has not been implemented at this time. Also, the performance of the current platform is unknown. This leads to the assignment at hand.

\subsection{Assignment}
Since the performance of the platform was not known, gathering data became the first task assigned to the team. Shortly after starting on the task, it was discovered that the entire project structure was unstable and that reliable data could not be retrieved. This lead to the next task, to improve the overal structure and stability of the platform.

One of the goals of the project was to put the platform on display at the Precision Fair\footnote{http://www.precisiebeurs.nl/} of 2013. To accomplish this, the team was required to set up demo machines, capable of displaying how the platform operated.

After the task of improving the structure and stability of the platform was completed, gathering performance data once again became possible, however, at this point the decision was made to create a simulation for the platform, allowing for automated testing and the gathering of performance data.

\subsection{Team}

\begin{table}[htbp]

\centering
\normalsize
\begin{tabular}{|l|l|}
\hline
Name & Role\\
\hline
E. Puik & Founding father\\
\hline
D. Telgen & Project supervisor\\
\hline
A. Streng & Team supervisor, researcher/software developer and scrum master\\
\hline
D. Jenkins & researcher/software developer\\
\hline
R. Scheefhals & researcher/software developer\\
\hline
\end{tabular}
\caption{Project team}
\end{table}
The researchers involved in the project are students at the university 
in Utrecht. They work directly under D. Telgen in a researching capacity.

\section{Organization}
\subsection{Knowledge Center for Technology and Innovation}
The research center forms a bridge across the requirements of professional practice and the research and education activities of the HU University of Applied Sciences Utrecht and is an innovation lab for small- and medium sized enterprises in the Utrecht region.

The research center is able to undertake projects and find solutions at low costs by using expertise available within higher education. There is also the opportunity to work with other knowledge institutes to find solutions for complex issues.

Projects and research are conducted under the guidance of professors and are carried out with the help of lecturers and students from the Bachelor studies. Collaboration with the business world is an important driver for the lecturers’ research and contributes to the innovation capacity in the region, where, by mutual agreement, research results are disseminated widely.

\subsection{Lectureship Microsystemtechnology/Embedded Systems}
The lectureship focuses on product improvement and the industrialisation of microsystems. It is located within the knowledge center for technology and innovation located at Oudenoord 700 Utrecht.

Since the year 2001, the function of lecturer was introduced into higher education. Lecturers guide groups of researchers, usually composed of teachers, which do practical research. One of these groups of researchers is named a lectureship. Practical research is aimed at solving problems which occur for professionals in the field. The research projects of this lectureship involves companies, students and tutors and addresses a current problem for a company.

\end{document}