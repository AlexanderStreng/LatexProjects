\documentclass[11pt]{report}
\begin{document}

\section{Project}
\subsection{Background}
In 2008 a project called HUniversal Production was started by E. Puik. This project was started to research agile manufacturing\cite{AGILEMAN} and to develop a working platform based on this manufacturing paradigm. Since then, the research into agile manufacturing has become larger and a lot more research was needed on different parts of the whole platform.
 
The first prototype of the platform consisted of a single machine capable of picking up and placing different kind of objects. While developing this first prototype a lot of focus was on the architecture of the platform. The architecture was based on a hybrid architecture consisting of ROS and JADE.
 
In 2012, the idea of reconfigurable equiplets was introduced. Being able to safely remove, replace or add new modules to a running system. This led to the development of MAST, or MAchine STates. Using MAST simplifies checking whether a module is busy or idle, but more importantly, it allows modules to be put in a safe state, making removal possible.
 
Currently, the platform uses MAS and ROS, which communicate through blackboards to create an agile manufacturing system. Reconfigurability has not been implemented at this time. Also, the performance of the current platform is unknown. This leads to the assignment at hand. More about the actual assignment will be explained later in this document.
\subsection{Assignment}
\subsection{Team}

\begin{table}[htbp]

\centering
\normalsize
\begin{tabular}{|l|l|}
\hline
Name & Role\\
\hline
E. Puik & Founding father\\
\hline
D. Telgen & Project supervisor\\
\hline
A. Streng & Team supervisor, researcher/software developer and scrum master\\
\hline
D. Jenkins & researcher/software developer\\
\hline
R. Scheefhals & researcher/software developer\\
\hline
\end{tabular}
\caption{Project team}
\end{table}
The researchers involved in the project are students at the university 
in Utrecht. They work directly under D. Telgen in a researching capacity.

\section{Organization}
\subsection{Lectoraat}
Knowledge Center for Technology and Innovation at
Oudenoord 700 in Utrecht.

\end{document}