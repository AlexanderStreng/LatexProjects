\documentclass[11pt]{report}
\begin{document}

\section{Background}
 
In 2008 a project called HUniversal Production was started by E. Puik. This project was started to research agile manufacturing\cite{AGILEMAN} and to develop a working platform based on this manufacturing paradigm. Since then, the research into agile manufacturing became larger and a lot more research was needed on different parts of the whole platform.
 
The first prototype of the platform consisted of a single machine capable of picking up and placing different kind of objects. While developing this first prototype a lot of focus was in the architecture of the platform. The architecture was based on a hybrid architecture consisting of ROS and JADE.
 
In 2012, the idea of reconfigurable equiplets was introduced. Being able to safely remove, replace or add new modules to a running system. This led to the development of MAST, or MAchine STates. Using MAST simplifies checking whether a module is busy or idle, but more importantly, it allows modules to be put in a safe state, making removal possible.
 
Currently, the platform uses MAS and ROS, which communicate through blackboards to create an agile manufacturing system. Reconfigurability has not currently been implemented. Also, the performance of the current platform is unknown. This leads to the assignment at hand. More about the actual assignment will be explained later in this document.
 
 
 \section{The organization}
%The University of Applied Sciences Utrecht is one of a few organizations involved in the project, the others are Avans in Breda, which will create their own features for the equiplet using source code from the university in Utrecht, Fontys, who created the electronics hardware of the ROSCAM, TU/E (Technical University Eindhoven) and NHL in Leeuwarden, who are involved in developing for the vision parts of the equiplet.
 
The organization through which this assignment was offered is the University of Applied Sciences Utrecht (HU). More specifically, the department for Microsystemtechnology and Embedded Systems (MST). This department involves itself with product-improvement and the industrialization of microsystems and can be found in the Knowledge Center for Technology and Innovation at Oudenoord 700 in Utrecht.
 
In this department, teachers, also known as lectors, work together with research teams to conduct research at a practical level. Their goal is to provide professionals with solutions when they run into problems in the field. The area of expertise for the MST department focuses on high-precision systems.
 
Research within the HU in general targets innovation and is closely connected to the education provided. Important aspects are knowledge and the distribution of knowledge. This is why the HU conducts research on a practical level.

\end{document}