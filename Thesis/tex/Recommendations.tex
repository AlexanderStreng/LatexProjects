\documentclass{../local}
\begin{document}

\section{Recommendations}
\subsection{Schedule Lock}
The schedule lock design described in section \ref{sec:schedulelock} (locked scheduling), is a proposed solution based on the dangers of multithreaded software. The design ensures that every product gets his time to plan his steps at the required equiplets. Also using this solution, when a deadlock occurs, it is possible to let certain products with a high priority schedule first when the product agents are negotiating to determine who goes first. However, It is possible to abandon the idea of using a lock for a schedule and use a 'first come, first served' design (free scheduling) so every product agent can try to schedule at the same time.

The idea of locked scheduling is based on the rule that the data equiplets' schedules may not change when a product agent is planning and scheduling at the corresponding equiplets (i.e. an atomic action). To create an optimal plan, the product agent needs the schedules of all of the equiplets that can perform the needed product steps. In big grids with multiple equiplets for a single capability, the result of the plan will only include some of the equiplets mentioned before. So some didn't had its schedule changed and could have been used by another product agent without violating the 'atomic scheduling' rule.

When not using locks for equiplet's schedules, this can be utilized. But using this kind of approach can be dangerous to the integrity of the plan created by the product agent, but can be resolved. Given the following situation with two product agents: PA$_1$ and PA$_2$, the following situation can occur. PA$_1$ has gathered all of the required data and has started calculating his optimal production path. PA$_2$ does the same, at the same time, and the result is the same as that of PA$_1$, but is done earlier. After PA$_2$ has pushed his schedule to the equiplets earlier, and used the equiplets that PA$_1$ was going to use too. PA$_1$'s calculated schedule has been compromised and will be detected when pushing his new schedule. This can be easily resolved by just restarting calculating the plan.

Advantage is that multiple product agents can use the same equiplets to calculate the plans and that these plans don't interfere with eachother. But this solution can be prone to error too. For example, having a product with a lot of product steps would require a relatively long schedule time, and a long batch with a very low amount of product steps. In the previously described situation, the batch products would cause the product to restart over and over again. This can result in a product not being able to produce.

However in real life situations, it is very unlikely that these cases will occur frequently. Using a simulation can prove if these cases will occur and if the free scheduling solution is feasible. It is possible that a combination of both is required.

\subsection{Scheduling algorithm improvements}
The current scheduling algorithm takes different aspects of the state of an equiplet into account. Choosing a suitable equiplet efficiently with the algorithm has been proven. What the algorithm doesn't take into account is optimizing transportation of products between equiplets.

When a product needs transportation to another equiplet for another product step to be excecuted, it is possible only taking the state of the equiplet into account can be misleading. For example, an equiplet that has a higher load could more efficient to produce at because its position is closer to the product's current whereabouts. Also the availability of transportation resources can be limited and therefore slowing down the production.

Thes algorithm needs to be improved by taking transportation components into account. This can be done by adding new formulas to the production matrix.

\subsection{Simulation improvements}
The simulation program was made in a very limited timeframe, therefor there are some features not implemented which would make the simulation work better. One of said features is to simulate a certain number of products rather than simulating a certain amount of time. This new feature makes it possible to run new kinds of test cases.

At this time, the scheduling algorthm has been implemented in the simulation. The REXOS platform does not have this scheduling algorithm, although a start has been made, the planning logic has not been finished. It is recommended that the scheduling algorithm in the simulation will be implemented in the REXOS platform. Best practice will be to make the scheduling algorithm usable by both systems while having a single codebase of the algorithm.

The graphical user interface could be extended to make it possible to modify all the parameters in the simulation, this would improve the user experience enormously. Another feature that would improve user experience is adding he possibility of outputting graphs for statistical data, rather than csv (comma-seperated value) files. In the current simulation, users have to manually create these graphs and this is very time consuming.

\end{document}