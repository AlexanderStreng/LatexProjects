\documentclass[11pt]{report}
\begin{document}

In this chapter we will describe the results gained in this project. We will tell something about gaining the the stable architecture, and the build up to the precision fair. We will also try to explain the results gained from research and simulation of the heterarchical vs. hierarchical system.

\section{REXOS Architecture}
Throughout the whole project the main subject was constantly improving and modifying the current REXOS architecture. During the first part of the research a deadline was set to write and research feasable demo's for the precision fair. After the precision fair, a simulation had to be written in order to research the advantages of reserving batches (hierarchical control).

\subsection{Stable architecture}
%smth about how messy the architecture was, is, and will be.
Whilst we aimed to improve the architecture throughout the  

\subsection{Demo's}


\section{Hierarchical scheduling vs. Heterarchical scheduling}
In this section we will try to demonstrate and explain the results we got when testing the simulation and drawing conclusions about Hierarchical scheduling versus Heterarchical scheduling.

Hierarchical scheduling in this example is defined as having certain reservations concerning batches. This means that in a grid, certain equiplets can be reserverd on a higher level (thus implying a hierarchy). Heterarchical scheduling allows for individualistic scheduling (i.e. no Interference from higher levels). 
	\subsection{Setup of the example case}
	In this sample test case we try to demonstrate the various advantages and disadvantages hierarchical scheduling can have over normal heterarchical scheduling.  Next we discuss how switching between heterarchical  and hierarchical can negotiate the disadvantages.
	Throughout this whole case we will use the same products, capabilities and grid - layout unless stated otherwise. The capabilities are as follows:\\
\begin{tabular}{ | l | l |}
\hline
\textbf{Name} & \textbf{Timeslots}  \\ \hline
P & 5  \\ \hline
A & 20 \\ \hline
S & 10 \\ \hline
\end{tabular}\\
\\
The products used are defined next. Throughout testing the minimal deadline is defined in this setup as 50 seconds. This is when 40 products are spawned every 120 seconds in a 3 x 3 grid without reservations as defined below. If the deadline is set to a number smaller than 50 seconds, products start to fail. In order to allow for a little bit of delay, a deadline of 60 seconds is set.\\
\\
\begin{tabular}{ | l | l |}
\hline
\textbf{Product name} & Test\_Product  \\ \hline
\textbf{Sequence of product steps} & P A P S \\ \hline
\textbf{deadline} & 00:00:60 ( 60 seconds ) \\ \hline
\end{tabular}\\
\\
The batch products are defined to be the same as the normal products in the grid. This is defined in a 3 x 3 grid where equiplet 1, 2 \& 3 are reserved for the batch. Through testing the minimal deadline for batches is defined on 56 seconds. In order to allow for a little bit of delay the deadlines are set to 60 seconds.\\
\\
\begin{tabular}{ | l | l |}
\hline
\textbf{Batch product name} & Test\_Product  \\ \hline
\textbf{Sequence of product steps} & P A P S \\ \hline
\textbf{deadline} & 00:00:60 ( 60 seconds ) \\ \hline
\end{tabular}\\
\\
In this case we define a grid setup of 3 x 3. Default travel distance between equiplets is set to 2 seconds. Throughout the case we will change some properties of the equiplets, let some throw some errors etc.  Now consider the following grid;\\
\\
\begin{tabular}{ | l | l | l |}
\hline
Equiplet 1 - capability(A) & Equiplet 2 - capability(S) & Equiplet 3 - capability(P)  \\ \hline
Equiplet 4 - capability(A) & Equiplet 5 - capability(P) & Equiplet 6 - capability(S) \\ \hline
Equiplet 7 - capability(A) & Equiplet 8 - capability(S) & Equiplet 9 - capability(P) \\ \hline
\end{tabular}\\
\\
In order to define travel times within the grid, a schedulematrix is constructed. The formula for generating this matrix is as follows:\\
\\
$(abs(A x - B x) + abs(A y - B y) ) \times 20.0$\\
\\
which results in the following travel matrix:
\\
\begin{tabular}{ | l | l | l | l | l | l | l | l | l | l |}
\hline
    & EQ1 & EQ2 & EQ3 & EQ4 & EQ5 & EQ6 & EQ7 & EQ8 & EQ9  \\ \hline
EQ1 & 0 & 20 & 40 & 20 & 40 & 60 & 40 & 60 & 80\\ \hline
EQ2 & 20 & 0 & 20 & 40 & 20 & 40 & 60 & 40 & 60\\ \hline
EQ3 & 40 & 20 & 0 & 60 & 40 & 20 & 80 & 60 & 40\\ \hline
EQ4 & 20 & 40 & 60 & 0 & 20 & 40 & 20 & 40 & 60\\ \hline
EQ5 & 40 & 20 & 40 & 20 & 0 & 20 & 40 & 20 & 40\\ \hline
EQ6 & 60 & 40 & 20 & 40 & 20 & 0 & 60 & 40 & 20\\ \hline
EQ7 & 40 & 60 & 80 & 20 & 40 & 60 & 0 & 20 & 40\\ \hline
EQ8 & 60 & 40 & 60 & 40 & 20 & 40 & 20 & 0 & 20\\ \hline
EQ9 & 80 & 60 & 40 & 60 & 40 & 20 & 40 & 20 & 0\\ \hline
\end{tabular}\\
\\

The time given in the travel matrix is the amount of timeslots. So in this case, when travelling from equiplet 1 to equiplet 2, the amount of time required is 20 * (timeslotlength(100ms)) 100 = 2000 ms travel time.

\subsection{Batch scheduling}

\subsection{Advantages of switching back to heterarchical scheduling}

\end{document}