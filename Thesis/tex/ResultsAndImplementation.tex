\documentclass{../local}
\begin{document}

In this chapter we will describe the results gained in this project. We will tell something about gaining the the stable architecture, and the build up to the precision fair. We will also try to explain the results gained from research and simulation of the heterarchical vs. hierarchical system.

\section{REXOS Architecture}
Throughout the whole project the main subject was constantly improving and modifying the current REXOS architecture. During the first part of the research a deadline was set to write and research feasable demo's for the precision fair. After the precision fair, a simulation had to be written in order to research the advantages of reserving batches (hierarchical control).

\subsection{Stable architecture - TODO}
%smth about how messy the architecture was, is, and will be.

\subsubsection{Demo's - TODO}


\section{Hierarchical scheduling vs. Heterarchical scheduling}

By utilizing the previously designed and implemented tools and components of REXOS, progress on the research question "Is REXOS best suited for a hierarchical or a heterarchical setup?" can be made. Scheduling products in the is one of the main components in REXOS. A lot of optimization can be reached on this matter. 

By using the simulation with implemented scheduling algorithm, hierarchical scheduling versus heterarchical scheduling can be tested. In this section, the execution and results of these tests are described.

Hierarchical scheduling in the following test cases is defined as having certain reservations concerning batches. This means that in a grid, certain equiplets can be reserverd on a higher level (thus implying a hierarchy) for the batch. Heterarchical scheduling allows for individualistic scheduling (i.e. no Interference from higher levels) and thus the batch will go through the grid alongside other products.

\subsection{Setup of the example case}
In this sample test case we try to demonstrate the various advantages and disadvantages hierarchical scheduling can have over normal heterarchical scheduling.  Next we discuss how switching between heterarchical  and hierarchical can negotiate the disadvantages.
	Throughout this whole case we will use the same products, capabilities and grid - layout unless stated otherwise. The capabilities are as follows:
	
\begin{center}
	\begin{tabular}{ | l | l |}
	\hline
	\textbf{Name} & \textbf{Timeslots}  \\ \hline
	P & 5  \\ \hline
	A & 20 \\ \hline
	S & 10 \\ \hline
	\end{tabular}
\end{center}

The products used are defined next. Throughout testing the minimal deadline is defined in this setup as 50 seconds. This is when 40 products are spawned every 120 seconds in a 3 x 3 grid without reservations as defined below. If the deadline is set to a number smaller than 50 seconds, products start to fail. In order to allow for a little bit of delay, a deadline of 60 seconds is set.

\begin{center}
	\begin{tabular}{ | l | l |}
	\hline
	\textbf{Product name} & Test\_Product  \\ \hline
	\textbf{Sequence of product steps} & P A P S \\ \hline
	\textbf{deadline} & 00:00:60 ( 60 seconds ) \\ \hline
	\end{tabular}
\end{center}

The batch products are defined to be the same as the normal products in the grid. This is defined in a 3 x 3 grid where equiplet 1, 2 \& 3 are reserved for the batch. Through testing the minimal deadline for batches is defined on 56 seconds. In order to allow for a little bit of delay the deadlines are set to 60 seconds.

\begin{center}
	\begin{tabular}{ | l | l |}
	\hline
	\textbf{Batch product name} & Test\_Product  \\ \hline
	\textbf{Sequence of product steps} & P A P S \\ \hline
	\textbf{deadline} & 00:00:60 ( 60 seconds ) \\ \hline
	\end{tabular}
\end{center}

In this case we define a grid setup of 3 x 3. Default travel distance between equiplets is set to 2 seconds. Throughout the case we will change some properties of the equiplets, let some throw some errors etc.  Now consider the following grid;

\begin{center}
	\begin{tabular}{ | l | l | l |}
	\hline
	Equiplet 1 - capability(A) & Equiplet 2 - capability(S) & Equiplet 3 - capability(P)  \\ \hline
	Equiplet 4 - capability(A) & Equiplet 5 - capability(P) & Equiplet 6 - capability(S) \\ \hline
	Equiplet 7 - capability(A) & Equiplet 8 - capability(S) & Equiplet 9 - capability(P) \\ \hline
	\end{tabular}
\end{center}

In order to define travel times within the grid, a schedulematrix is constructed. The formula for generating this matrix is as follows:

\begin{center}
	$(abs(A x - B x) + abs(A y - B y) ) \times 20.0$
\end{center}

which results in the following travel matrix:

\begin{center}
	\begin{tabular}{ | l | l | l | l | l | l | l | l | l | l |}
	\hline
		& EQ1 & EQ2 & EQ3 & EQ4 & EQ5 & EQ6 & EQ7 & EQ8 & EQ9  \\ \hline
	EQ1 & 0 & 20 & 40 & 20 & 40 & 60 & 40 & 60 & 80\\ \hline
	EQ2 & 20 & 0 & 20 & 40 & 20 & 40 & 60 & 40 & 60\\ \hline
	EQ3 & 40 & 20 & 0 & 60 & 40 & 20 & 80 & 60 & 40\\ \hline
	EQ4 & 20 & 40 & 60 & 0 & 20 & 40 & 20 & 40 & 60\\ \hline
	EQ5 & 40 & 20 & 40 & 20 & 0 & 20 & 40 & 20 & 40\\ \hline
	EQ6 & 60 & 40 & 20 & 40 & 20 & 0 & 60 & 40 & 20\\ \hline
	EQ7 & 40 & 60 & 80 & 20 & 40 & 60 & 0 & 20 & 40\\ \hline
	EQ8 & 60 & 40 & 60 & 40 & 20 & 40 & 20 & 0 & 20\\ \hline
	EQ9 & 80 & 60 & 40 & 60 & 40 & 20 & 40 & 20 & 0\\ \hline
	\end{tabular}
\end{center}

The time given in the travel matrix is the amount of timeslots. So in this case, when travelling from equiplet 1 to equiplet 2, the amount of time required is 20 * (timeslotlength(100ms)) 100 = 2000 ms travel time.

\subsection{Batch scheduling - TODO}

\subsection{Advantages of switching back to heterarchical scheduling}
As defined in previous paragraph, batch scheduling can be advantagous over heterarchical scheduling. In this section we explain the downsides of batch scheduling, and try pose a possible solution. 


\subsubsection*{Equiplet errors}
One of the major downsides concerning reserved equiplets is what happends when one of those equiplets experiences an error. If the amount of reserved equiplets is small, products may start failing. In the next case a constant load of 80\% is maintained. We are going to throw some errors at equiplet 3 ( capability P ).

\begin{center}
	\begin{tabular}{ | l | l |}
	\hline
	\textbf{Simulation time} & 2 hours \\ \hline
	\textbf{Maintained load} & 80\% \\ \hline
	\textbf{Batch spawn settings} & 1 product each 2 seconds \\ \hline
	\textbf{Equiplets reserved for batch} & equiplet 1, equiplet 2 \& equiplet 3  \\ \hline
	\textbf{Equiplets in error} & equiplet 3 \\ \hline
	\textbf{Error interval, duration, start time} & After 1 hour, the equiplet experiences an error lasting 1 hour \\ \hline
	\end{tabular}
\end{center}

\begin{center}
	\begin{figure}[h!]
		\includegraphics[width=17cm]{../images/CombinedEquipletLoadNotAllowingOutsideReservation.png}
		\centering
		\caption{Combined load of all equiplets in the grid}
	\end{figure}
	\label{fig:CombinedEquipletLoadNotAllowingOutsideReservation}
\end{center}

Figure ~\ref{CombinedEquipletLoadNotAllowingOutsideReservation} demonstrates that after 1 hour equiplets 1, 2 \& 3 drop to 0\%  load. This is because of the fact that when equiplet 3 has an error, all the batch products fail. Because they cannot schedule somewhere else (they are only allowed to schedule at the reserved equiplets).
In the next case we are going to look at what happends to the grid as a total when the batches are allowed to switch back to heterarchical scheduling. The same test case is used but this time the batch products are allowed to use the remaining equiplets in the grid as a 'backup'. The products can only use the remaining equiplet when one or more equiplets in the reserved line are in error.
Lets see what happends to the combined equiplet load;

\begin{center}
	\begin{figure}[h!]
		\includegraphics[width=17cm]{../images/CombinedEquipletLoadAllowingOutsideReservation.png}
		\centering
		\caption{Combined load of all equiplets in the grid}
	\end{figure}
	\label{fig:CombinedEquipletLoadAllowingOutsideReservation}
\end{center}

As shown in figure ~\ref{CombinedEquipletLoadAllowingOutsideReservation} around the 1 hour mark, right when equiplet 3 has its error, a small decline in load is visible. Right afterwards the reserved equiplets 1 \& 2 begin producing again. This is because the products are allowed to enter the grid for their needs. If we compare the load of all equiplets with the same capability, its becomes apparent what happends;

\begin{center}
	\begin{figure}[h!]
		\includegraphics[width=17cm]{../images/CapabilityEquipletLoadNotAllowingOutsideReservation.png}
		\centering
		\caption{Load of all equiplets with capability P (allowing scheduling outside reserved equiplets)}
	\end{figure}
	\label{fig:CapabilityEquipletLoadNotAllowingOutsideReservation}
\end{center}
in figure ~\ref{CapabilityEquipletLoadNotAllowingOutsideReservation} the load of all equiplets with capability P is displayed. When equiplet 3 breaks down, equiplet 5 \& 9 continue producing in the same pace.

\begin{center}
	\begin{figure}[h!]
		\includegraphics[width=17cm]{../images/CapabilityEquipletLoadAllowingOutsideReservation.png}
		\centering
		\caption{Load of all equiplets with capability P (not allowing scheduling outside reserved equiplets)}
	\end{figure}
	\label{fig:CapabilityEquipletLoadAllowingOutsideReservation}
\end{center}
in figure ~\ref{CapabilityEquipletLoadNotAllowingOutsideReservation} the load of all equiplets with capability P is displayed. When equiplet 3 breaks down and all the batch products enter the 'un-reserved' part of the grid, equiplets 5 \& 9 take over.

\end{document}
