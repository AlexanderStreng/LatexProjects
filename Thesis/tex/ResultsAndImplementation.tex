\documentclass{../local}
\begin{document}

In this chapter we will describe the results gained in this project. We will tell something about gaining the the stable architecture, and the build up to the precision fair. We will also try to explain the results gained from research and simulation of the heterarchical vs. hierarchical system.

\section{REXOS Architecture}
Throughout the whole project the main subject was constantly improving and modifying the current REXOS architecture. During the first part of the research a deadline was set to research and write feasable demo's for the precision fair. After the precision fair, a simulation had to be written in order to research the advantages of reserving batches (hierarchical control).

\subsection{Stable architecture - TODO}
Having a stable architecture is an important aspect of a good platform. During the project, several key issues were identified and dealt with. These key issues, and how they were tackled will be explained in more detail below.

\subsubsection{Agent communication}
%implementation of the 'FIPA compliant' way of communicating
One of the first things that was discovered during the project was that the communication of the agents in the system was chaotic and inconsistent. This was discovered when a mapping was made of all the messages sent by the agents. This mapping showed that the incorrect ontologies were being used, and that the ontologies were not consistent either, for example: a message sent from agent A with the ontology "IsEquipletBusy" to agent B, would be answered with a different ontology, such as "IsEquipletBusyConfirmed". This usage of different ontologies does not adhere to the FIPA Standard\footnote{http://www.fipa.org/}. The initial agent communication scheme can be found in section \ref{sec:AgentComm}

After the FIPA Standards were applied to this, the resulting communication scheme looked like the following:

\begin{center}
  \includegraphics[width=20cm]{../images/RFC-Ontology-Performatives.pdf}
  \captionof{figure}{Agent Communication after FIPA Standard was applied}
\end{center}

\subsubsection{Knowledge database}
The new design for the knowledge database, which can be found in section ~\ref{sec:kdb}, was implemented on an isolated system. This allowed the development of other systems to continue, while MAS was being re-written to use the new database design. The implementation of MAS was not finished however.

\subsubsection{REXOS vision}
During this semester the interns implemented REXOS vision. This means that they implemented a new vision system which works properly with all the components of the REXOS architecture. The vision system is implemented in ROS, which means it is implemented as a node. The vision node is module specific. By calibrating the camera, distortion and fish eye effect is removed (as much as possible).

\subsubsection*{Coordinate system}
As a by-product of developing a new vision architecture, a new coordinate system was introduced. This systems uses position based parameters to define a precise coordinate system. This way all modules, independend of their location can use a generic convertion to determine their position on the equiplet.


\section{Hierarchical scheduling vs. Heterarchical scheduling}

By utilizing the previously designed and implemented tools and components of REXOS, progress on the research question "Is REXOS best suited for a hierarchical or a heterarchical setup?" can be made. Scheduling products in the is one of the main components in REXOS. A lot of optimization can be reached on this matter. 

By using the simulation with implemented scheduling algorithm, hierarchical scheduling versus heterarchical scheduling can be tested. In this section, the execution and results of these tests are described.

Hierarchical scheduling in the following test cases is defined as having certain reservations concerning batches. This means that in a grid, certain equiplets can be reserverd on a higher level (thus implying a hierarchy) for the batch. Heterarchical scheduling allows for individualistic scheduling (i.e. no Interference from higher levels) and thus the batch will go through the grid alongside other products.

\subsection{Setup of the example case}
In this sample test case we try to demonstrate the various advantages and disadvantages hierarchical scheduling can have over normal heterarchical scheduling.  Next we discuss how switching between heterarchical  and hierarchical can negotiate the disadvantages.
	Throughout this whole case we will use the same products, capabilities and grid - layout unless stated otherwise. The capabilities are as follows:
	
\begin{center}
	\begin{tabular}{ | l | l |}
	\hline
	\textbf{Name} & \textbf{Timeslots}  \\ \hline
	P & 5  \\ \hline
	A & 20 \\ \hline
	S & 10 \\ \hline
	\end{tabular}
\end{center}

The products used are defined next. Throughout testing the minimal deadline is defined in this setup as 50 seconds. This is when 40 products are spawned every 120 seconds in a 3 x 3 grid without reservations as defined below. If the deadline is set to a number smaller than 50 seconds, products start to fail. In order to allow for a little bit of delay, a deadline of 60 seconds is set.

\begin{center}
	\begin{tabular}{ | l | l |}
	\hline
	\textbf{Product name} & Test\_Product  \\ \hline
	\textbf{Sequence of product steps} & P A P S \\ \hline
	\textbf{deadline} & 00:00:60 ( 60 seconds ) \\ \hline
	\end{tabular}
\end{center}

The batch products are defined to be the same as the normal products in the grid. This is defined in a 3 x 3 grid where equiplet 1, 2 \& 3 are reserved for the batch. Through testing the minimal deadline for batches is defined on 56 seconds. In order to allow for a little bit of delay the deadlines are set to 60 seconds.

\begin{center}
	\begin{tabular}{ | l | l |}
	\hline
	\textbf{Batch product name} & Test\_Batch  \\ \hline
	\textbf{Sequence of product steps} & P A P S \\ \hline
	\textbf{deadline} & 00:00:60 ( 60 seconds ) \\ \hline
	\end{tabular}
\end{center}

In this case we define a grid setup of 3 x 3. Default travel distance between equiplets is set to 2 seconds. Throughout the case we will change some properties of the equiplets, let some throw some errors etc.  Now consider the following grid;

\begin{center}
	\begin{tabular}{ | l | l | l |}
	\hline
	Equiplet 1 - capability(A) & Equiplet 2 - capability(S) & Equiplet 3 - capability(P)  \\ \hline
	Equiplet 4 - capability(A) & Equiplet 5 - capability(P) & Equiplet 6 - capability(S) \\ \hline
	Equiplet 7 - capability(A) & Equiplet 8 - capability(S) & Equiplet 9 - capability(P) \\ \hline
	\end{tabular}
\end{center}

In order to define travel times within the grid, a schedulematrix is constructed. The formula for generating this matrix is as follows:

\begin{center}
	$(abs(A x - B x) + abs(A y - B y) ) \times 20.0$
\end{center}

which results in the following travel matrix:

\begin{center}
	\begin{tabular}{ | l | l | l | l | l | l | l | l | l | l |}
	\hline
		& EQ1 & EQ2 & EQ3 & EQ4 & EQ5 & EQ6 & EQ7 & EQ8 & EQ9  \\ \hline
	EQ1 & 0 & 20 & 40 & 20 & 40 & 60 & 40 & 60 & 80\\ \hline
	EQ2 & 20 & 0 & 20 & 40 & 20 & 40 & 60 & 40 & 60\\ \hline
	EQ3 & 40 & 20 & 0 & 60 & 40 & 20 & 80 & 60 & 40\\ \hline
	EQ4 & 20 & 40 & 60 & 0 & 20 & 40 & 20 & 40 & 60\\ \hline
	EQ5 & 40 & 20 & 40 & 20 & 0 & 20 & 40 & 20 & 40\\ \hline
	EQ6 & 60 & 40 & 20 & 40 & 20 & 0 & 60 & 40 & 20\\ \hline
	EQ7 & 40 & 60 & 80 & 20 & 40 & 60 & 0 & 20 & 40\\ \hline
	EQ8 & 60 & 40 & 60 & 40 & 20 & 40 & 20 & 0 & 20\\ \hline
	EQ9 & 80 & 60 & 40 & 60 & 40 & 20 & 40 & 20 & 0\\ \hline
	\end{tabular}
\end{center}

The time given in the travel matrix is the amount of timeslots. So in this case, when travelling from equiplet 1 to equiplet 2, the amount of time required is 20 * (timeslotlength(100ms)) 100 = 2000 ms travel time.

\subsection{Batch scheduling}
In this section, the differences between heterarchical scheduling and hierarchical scheduling ( reservating equiplets for batch ) will be described. This will be done by describing a case and using the simulation to gather data. For this simulation the equiplet and capability setup will be the same as in the example case. The simulation will run for 2 hours. The following batch product will be used:

\begin{center}
	\begin{tabular}{ | l | l |}
	\hline
	\textbf{Batch product name} & Batch1  \\ \hline
	\textbf{Sequence of product steps} & P A P S \\ \hline
	\textbf{Deadline} & 00:01:26 ( 86 seconds ) \\ \hline
	\textbf{Spawn interval} & 1 product per 2 seconds \\ \hline
	\end{tabular}
\end{center}

For the hierarchical setup, Equiplets 1, 2 and 3 will be reserved for the batch. For the products that will run on the rest of the grid, a dynamic product spawner will be used to simulate real life situations. This product spawner will spawn 35000 products of which are 3600 products of Batch1. These amount of products are chosen so that the grid will perform at approximately 80\% load.

In the results, we will look at the production efficiency of the equiplets. Concerning the products, we will look at throughput, failed products and amount of products simultaneously in the grid.

\subsubsection*{Heterarchical simulation}

\begin{center}
	\begin{figure}[h!]
		\includegraphics[width=17cm]{../images/BatchSchedulingCurrentLoadHeterarchical.png}
		\centering
		\caption{Current load of equiplets - Heterarchical}
	\end{figure}
	\label{fig:BatchSchedulingCurrentLoadHeterarchical}
\end{center}

Figure ~\ref{BatchSchedulingCurrentLoadHeterarchical} shows a high load on all the equiplets. The equiplets with the longest capability duration takes the longest. Equiplet 9 is most of the time not needed for all of the products, but needs to help out to be able to produce all the products. The overall average current load of these equiplets is 77\%

\subsubsection*{Hierarchical simulation}

\begin{center}
	\begin{figure}[h!]
		\includegraphics[width=17cm]{../images/BatchSchedulingCurrentLoadHierarchical.png}
		\centering
		\caption{Current load of equiplets - Hierarchical}
	\end{figure}
	\label{fig:BatchSchedulingCurrentLoadHierarchical}
\end{center}

Figure ~\ref{BatchSchedulingCurrentLoadHierarchical} shows a high load on all of the equiplets. Equiplets 1, 2 and 3 have a constant load, which is to be expected since they are a optimized setup and the batch can only access these equiplets. The remaining equiplets have a high load, but is lower than the load of the Heterarchical setup. The average current load of this grid is 69\%

\newpage
\subsubsection*{Products}

	\includegraphics[width=16cm]{../images/BatchSchedulinThroughputFailedProducts.png}
	\captionof{figure}{Failed products - Created products}
	\begin{center}
	\includegraphics[width=8cm]{../images/BatchSchedulingProductsSimultaneouslyInGrid.png}
	\captionof{figure}{products being produced simultaneously}
	\end{center}
%\caption{Current load of equiplets - Hierarchical}
%\label{fig:BatchSchedulingCurrentLoadHierarchical}

As seen in above graphs, heterarchical is more production efficient. It has a higher throughput, but uses a lot more resources and some products could not finish. The amount of simultaneous products in the grid is more than double than at the hierarchical setup. Also the heterarchical setup has failed products, whereas the hierarchical setup has none. The amount of failed products is 4\% relative to the throughput.

\subsection{Advantages of switching back to heterarchical scheduling}
As defined in previous paragraph, batch scheduling can be advantagous over heterarchical scheduling. In this section we explain the downsides of batch scheduling, and try pose a possible solution. 


\subsubsection*{Equiplet errors}
One of the major downsides concerning reserved equiplets is what happends when one of those equiplets experiences an error. If the amount of reserved equiplets is small, products may start failing. In the next case a constant load of 80\% is maintained. We are going to throw some errors at equiplet 3 ( capability P ).

\begin{center}
	\begin{tabular}{ | l | l |}
	\hline
	\textbf{Simulation time} & 2 hours \\ \hline
	\textbf{Maintained load} & 80\% \\ \hline
	\textbf{Batch spawn settings} & 1 product each 2 seconds \\ \hline
	\textbf{Equiplets reserved for batch} & equiplet 1, equiplet 2 \& equiplet 3  \\ \hline
	\textbf{Equiplets in error} & equiplet 3 \\ \hline
	\textbf{Error interval, duration, start time} & After 1 hour, the equiplet experiences an error lasting 1 hour \\ \hline
	\end{tabular}
\end{center}

\begin{center}
		\includegraphics[width=17cm]{../images/CombinedEquipletLoadNotAllowingOutsideReservation.png}
		\centering
		\captionof{figure}{Combined load of all equiplets in the grid}
	\label{fig:CombinedEquipletLoadNotAllowingOutsideReservation}
\end{center}

Figure ~\ref{CombinedEquipletLoadNotAllowingOutsideReservation} demonstrates that after 1 hour equiplets 1, 2 \& 3 drop to 0\%  load. This is because of the fact that when equiplet 3 has an error, all the batch products fail. Because they cannot schedule somewhere else (they are only allowed to schedule at the reserved equiplets).
In the next case we are going to look at what happends to the grid as a total when the batches are allowed to switch back to heterarchical scheduling. The same test case is used but this time the batch products are allowed to use the remaining equiplets in the grid as a 'backup'. The products can only use the remaining equiplet when one or more equiplets in the reserved line are in error.
Lets see what happends to the combined equiplet load;

\begin{center}
		\includegraphics[width=17cm]{../images/CombinedEquipletLoadAllowingOutsideReservation.png}
		\centering
		\captionof{figure}{Combined load of all equiplets in the grid}
	\label{fig:CombinedEquipletLoadAllowingOutsideReservation}
\end{center}

As shown in figure ~\ref{CombinedEquipletLoadAllowingOutsideReservation} around the 1 hour mark, right when equiplet 3 has its error, a small decline in load is visible. Right afterwards the reserved equiplets 1 \& 2 begin producing again. This is because the products are allowed to enter the grid for their needs. If we compare the load of all equiplets with the same capability, its becomes apparent what happends;

\begin{center}
		\includegraphics[width=17cm]{../images/CapabilityEquipletLoadNotAllowingOutsideReservation.png}
		\centering
		\captionof{figure}{Load of all equiplets with capability P (allowing scheduling outside reserved equiplets)}
	\label{fig:CapabilityEquipletLoadNotAllowingOutsideReservation}
\end{center}
in figure ~\ref{CapabilityEquipletLoadNotAllowingOutsideReservation} the load of all equiplets with capability P is displayed. When equiplet 3 breaks down, equiplet 5 \& 9 continue producing in the same pace.

\begin{center}
		\includegraphics[width=17cm]{../images/CapabilityEquipletLoadAllowingOutsideReservation.png}
		\centering
		\captionof{figure}{Load of all equiplets with capability P (not allowing scheduling outside reserved equiplets)}
	\label{fig:CapabilityEquipletLoadAllowingOutsideReservation}
\end{center}
in figure ~\ref{CapabilityEquipletLoadNotAllowingOutsideReservation} the load of all equiplets with capability P is displayed. When equiplet 3 breaks down and all the batch products enter the 'un-reserved' part of the grid, equiplets 5 \& 9 take over.

\section{Simulation}
\subsection{Gui}
\label{sec:di-gui}
The main simulation interface was implemented by another team, which is why this section will only cover the simulation editor interface.

Since the goal of the interface is creating virtual objects that represent the objects on the platform, it was decided to create several classes which describe these objects, rather than using the data classes already available. The existing data classes expect certain parameters which are unobtainable while editing the simulation. Due to this restriction, the following descriptive classes were implemented:

\begin{itemize}
\item EquipletDescription
\item BatchDescription
\item ProductDescription
\end{itemize}

Another goal of the interface is to output the created objects in a particular format, to be used in the simulation itself. The resulting format can be found in the following figures.

\begin{center}
	\includegraphics[width=7cm]{../images/capabilityFormat.png}
	\captionof{figure}{Capability Format}
\end{center}

The figure above shows the format in which capabilities are exported. It's a comma-seperated values file (csv), in which the first field represents the name of a capability, and where the second field represents how long a step of this type would take, noted in timeslots.

\begin{center}
	\includegraphics[width=15cm]{../images/productFormat.png}
	\captionof{figure}{Product Format}
\end{center}

This figure shows the product format used for the simulation. On the left side of the file, the fields are listed, while on the right side an example is shown. The example product is a smoothie. It has 30 minutes to be prepared and it's made by adding apples, oranges, bananas and ice, followed by blending these ingredients.

\begin{center}
	\includegraphics[width=10cm]{../images/batchFormat.png}
	\captionof{figure}{Batch Format}
\end{center}

This figure shows the format used for batches. The format is very similar to the one used for products. The amount indicates how many of these products should be made, and the interval specifies the time in seconds before another batch of products is started.

\begin{center}
	\includegraphics[width=13cm]{../images/equipletLayoutFormat.png}
	\captionof{figure}{Equiplet Layout Format}
\end{center}

This figure shows the format used to save the equiplet layout (the layout of equiplets within a grid). Instead of a csv format or a regular text format, this file is coded in JSON. Each equiplet has a name, an array of capability names, specifies whether the equiplet is reserved for a particular batch id number and has an array of optional equiplet error entries. These equiplet errors are defined as being a hardware or software error, how often they appear and for how long, and also whether the product currently being made on the equiplet breaks due to the error.

The file further specifies the grid, which is an array of arrays, each subarray represents a row of equiplets. Which means this example defines a grid of two rows of each two equiplets that looks like the following:

\begin{center}
	\includegraphics[width=8cm]{../images/Grid.png}
	\captionof{figure}{Example Grid Layout}
\end{center}



\end{document}
