\documentclass{../local}
\begin{document}
 % Summarize what the project is, what our assignments were, the results of our research & other stuff, conclusion

\section{Summary}
This thesis describes research done for the improvement of the REXOS project. 

Most current day manufacturing systems consist of large productions lines capable of producing a single unchanging product at high speeds. This kind of manufacturing is static manufacturing. In order to overcome the downsides of static manufacturing, research is being done to develop new paradigms. One of these new paradigms is what Y. Koren describes as \emph{'Reconfigurable machines (RMS) are machines whose structures can be changed to provide alternative functionality or/and upgradeable capacity on demand'}\cite{Koren}

The University of Applied Sciences Utrecht formulated their own solution regarding manufacturing. The REXOS project is this solution. REXOS stands for Reconfigurable equiplets operating system, or REQs-OS. REXOS uses the concept of grid manufacturing, which is based on RMS. REXOS consists of two main layers, a MAS layer and a ROS layer. The MAS handles all the cognitive decisions, while ROS executes the commands given to it by MAS.

This project is in its second iteration. The first iteration oversaw the first implementation of the MAS. Certain elements have not been (fully) implemented and or researched yet. The next paragraphs describe a number of these elements, followed by a paragraph for the conclusions.

When this iteration of the project started, the architecture of the project was unstable and it could not run without generating errors. One of the issues discovered while improving the architecture is that the communication between agents was confusing and inconsistent. The ontologies used in the communication were inconsistent and were not FIPA compliant.

One of the elements not implemented into the first iteration of the MAS was scheduling. A scheduling algorithm was researched however, and therefor that algorithm was discussed as a possible solution. There were two major issues with the algorithm however, and this resulted in a new implementation being researched using this algorithm as its base.

During the project, a simulation of the platform was created in order to gather data for the FAIM paper found in appendix~\ref{app:faim}. This simulation enabled the running of test scenarios or cases. The new implementation for the scheduling algorithm was implemented into the simulation and was tested during the running of the test cases. The test cases included scenarios with and without batch reservations, exclusive and non exclusive.

During the project, the conclusion was formed that the previously researched scheduling algorithm was partially usable in the project, but required adjustments. Further, another conclusion was formed that the optimal architecture for REXOS is a hybrid between hierarchical and heterarchical. Starting in a hierarchical fashion and switching to heterarchical in the event that an error occurs.

\end{document}