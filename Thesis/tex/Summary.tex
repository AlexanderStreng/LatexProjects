\documentclass{../local}
\begin{document}
 % Summarize what the project is, what our assignments were, the results of our research & other stuff, conclusion

\section{Summary}
This thesis describes research done for the improvement of the REXOS project. 

Most current day manufacturing systems consist of large productions lines capable of producing a single kind of product at high speeds. This type of manufacturing is called static manufacturing. In order to overcome the disadvantages of static manufacturing, research is being done to develop new paradigms. One of these new paradigms is what Y. Koren describes as \emph{"Reconfigurable machines (RMS) are machines whose structures can be changed to provide alternative functionality or/and upgradeable capacity on demand"}\cite{Koren}

The University of Applied Sciences Utrecht formulated their own solution regarding manufacturing. The REXOS project is this solution. REXOS uses the concept of grid manufacturing, which is based on RMS. 

When this iteration of the project started, the architecture of the project was unstable and it could not run without generating errors. The architecture had several flaws in its implementation. Such as inconsistent communication flows, an incomplete scheduling algorithm and a equiplet node capable of supporting only one kind of module.

Solutions for these problems are made in the form of design and implementation. Some of these implementations are used to gain answers to certain research questions. Implementing these solutions resulted in improvements of the REXOS platform without needing to test certain aspects.

One of these solutions is a simulation of the platform. This was created in order to gather data for the FAIM paper found in appendix~\ref{app:faim}. This simulation enabled the running of test scenarios or cases. The new design for the scheduling algorithm was implemented into the simulation and was tested during the running of the test cases. The test cases included scenarios with and without batch reservations, exclusive and non exclusive reservation.

The project end results include a more stable architecture with a partially implemented scheduling algorithm. Whilst the REXOS architecture is likely to change in the future, the current situation is stable enough to build upon. A simulator was developed as well. The main reason for this simulator was testing types of control in REXOS and by doing so proof that REXOS could make good use of implementing switchable architectures is deliverd. 

\end{document}