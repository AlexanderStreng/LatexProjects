\documentclass[final]{beamer}
\mode<presentation>
{
  \usetheme{I6pd2}
}
\graphicspath{{figures/}} 

\usepackage{epstopdf}
\usepackage{times}
\usepackage{amsmath,amssymb}
\usepackage[english]{babel}
\usepackage[latin1]{inputenc}
\usepackage{ragged2e} 
\usepackage{multicol}
\usepackage{tabularx}
\usepackage{changepage}

\addtobeamertemplate{block begin}
  {}
  {
   \begin{adjustwidth}{1cm}{1cm}
}
\addtobeamertemplate{block end}
  {\end{adjustwidth}}% Pads bottom of block

\setbeamertemplate{bibliography item}[text]

\usepackage[orientation=portrait, size=a0, scale=1.4, debug]{beamerposter} 
\title[Fancy Posters]{Hierarchical vs. Heterarchical}
\author{Alexander Streng, Roy Scheefhals, Duncan Jenkins}

\institute{HU University of Applied Sciences Utrecht}
\date{\today}

\newcommand{\columnblock}[2][Title]{
    \begin{block}{\large \vspace{-0.5cm}#1}
    #2
    \end{block}
}

\newcommand{\twocolumnsblock}[3][Title]{
    \columnblock[#1]{
        \begin{multicols}{2}
            #2
            \newpage
            #3
        \end{multicols}
    }
}

\begin{document}


\begin{frame}{}
    \begin{columns}[t]   
		\begin{column}{.48\textwidth}
   			\columnblock[Introduction]{
    		\justifying 
Manufacturing paradigms such as Agile Manufacturing (AM) and Reconfigurable Machine Systems (RMS) revolve around flexibility \cite{KorenRMS}. The REXOS project utilizes a Grid Manufacturing (GM) setup, which is based on RMS \cite{GRIDPARADIGM}. In GM, different approaches can be used for scheduling, namely hierarchical and heterarchical. This poster shows the results of comparing the two approaches with a simulation program and explores the pros and cons considering both setups.
		} 
	
 			\columnblock[Hierarchical setup]{
            \justifying 
            \includegraphics[clip=true, width=\linewidth, height=17cm]{GridEquipletHier.jpg}\\
			In this setup a possible hierarchical grid is displayed. Capabilities are displayed as $\alpha$, $\beta$ \& $\gamma$. The top three equiplets are reserved for batch production.
		}
		\end{column}
		\begin{column}{.48\textwidth}
            \columnblock[Heterarchical setup]{
            \justifying 
			\includegraphics[clip=true, width=\linewidth, height=17cm]{GridEquipletHeter.jpg}\\
			In this setup a typical heterarchical grid is displayed. Capabilities are displayed as $\alpha$, $\beta$ \& $\gamma$. All equiplets are equal in this setup.
    		}
    		\columnblock[Simulation setup]{
			The simulation program simulates two hours of production time with 9 equiplets. One simulation is run with the heterarchical setup and one with the hierarchical setup. The simulation will be given a random amount of products and a batch. A batch is defined as a single type of product produced at a high volume. In the graphs below, EQ stands for Equiplet and the numbers correspond to the equiplets illustrated from top left to bottom right. Current load indicates the amount of work the equiplet has done at a specific time in percentages.
				\justifying 
			}
    
			\end{column}
    	\end{columns}
       	\begin{columns}[t]  
			\begin{column}{.48\textwidth}
				\columnblock[Hetarchical Equiplet Load]{
				\includegraphics[width=.98\linewidth]{BatchSchedulingCurrentLoadHeterarchical.png}\\
			The equiplets are producing almost continuously. Equiplet 9 is used every now and then, but is idle for most of the time.
				\justifying 
				}
            \end{column}
			\begin{column}{.48\textwidth}
			\columnblock[Hierarchical Equiplet Load]{
				\includegraphics[width=.98\linewidth]{BatchSchedulingCurrentLoadHierarchical.png}\\
			Production is distributed over all of the equiplets. Equiplet 1, 2 and 3 are producing continuously because they are working on a batch.
				\justifying 
				}
            \end{column}
    \end{columns}
    
        
       \begin{columns}[t]  
			\begin{column}{.3125\textwidth}
				\columnblock[Throughput]{
				\includegraphics[width=.98\linewidth]{BatchSchedulingThroughput.png}\\
			As seen in the graph above, using heterarchical scheduling results in a higher production throughput within a certain timeframe. Hierarchical scheduling has a lower throughput due to the fact that the equiplets which are reserved for batches cannot be utilized at their full potential.
				\justifying 
				}
            \end{column}
			\begin{column}{.3125\textwidth}
			\columnblock[Products in Grid]{
			\includegraphics[width=.98\linewidth]{BatchSchedulingProductsSimultaneouslyInGrid.png}\\
When using a hierarchical setup, products and batches will be placed in a more controlled environment resulting in lower transportation needs. Heterarchical has higher needs because the maximum transport distance is higher. The further products need to travel, the longer they remain in the grid.
				\justifying 
				}
            \end{column}
            \begin{column}{.3125\textwidth}
			\columnblock[Failed Products]{
			\includegraphics[width=.98\linewidth]{BatchSchedulingFailedProducts.png}\\
A heterarchical setup tends to be more chaotic than a hierarchical setup. Production is faster, but products will remain in the grid for longer periods of time causing them to have a higher chance of failing. These products fail because their deadlines expire, which is the result of long transport times.
				\justifying 
				}
            \end{column}
    \end{columns}
    
   	\begin{columns}[t]
			\begin{column}{.65\textwidth}
							\columnblock[Conclusion]{
							\justifying
Using a heterarchical scheduling setup, equiplets will have a higher throughput, but at the cost of transportation. Putting a high load on the grid increases the chances of a product failing. A heterarchical setup is recommended when there are a lot of different products that need to be produced simultaneously. Hierarchical scheduling however is more efficient when there are limited transportation resources or if there is a need to control production. Reserving equiplets for batches of products could be desired to make sure that batches finish in time. Reserving equiplets for batches also ensures there is less interference for the batch.
				}
            \end{column}
            \begin{column}{.3138\textwidth}
               \columnblock[References]{
				\justifying 
				
\begin{thebibliography}{9}
				\bibitem{KorenRMS}
				Y. Koren,
				\emph{General RMS Characteristics. Comparison with Dedicated and Flexible Systems}
				\bibitem{GRIDPARADIGM}
				Daniel Telgen et al,
				\emph{Agile Control Architecture for Reconfigurable Manufacturing Systems}
\end{thebibliography}

	}
	%, Research Centre Technology \& 					Innovation, HU University of Applied Sciences Utrecht
            \end{column}
    \end{columns}
\end{frame}
\end{document}