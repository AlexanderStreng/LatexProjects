\documentclass[final]{beamer}
\mode<presentation>
{
  \usetheme{I6pd2}
}
\graphicspath{{figures/}} % e.g. where you have your logos

\usepackage{epstopdf}
\usepackage{times}
\usepackage{amsmath,amssymb}
\usepackage[english]{babel}
\usepackage[latin1]{inputenc}
\usepackage{ragged2e} 
\usepackage{multicol}
\usepackage{tabularx}
\usepackage{changepage}

\addtobeamertemplate{block begin}
  {}
  {
   \begin{adjustwidth}{1cm}{1cm}
}
\addtobeamertemplate{block end}
  {\end{adjustwidth}}% Pads bottom of block

\setbeamertemplate{bibliography item}[text]

\usepackage[orientation=portrait, size=a0, scale=1.4, debug]{beamerposter} 
\title[Fancy Posters]{REXOS}
\author{Daniel Telgen et al.}

\institute{HU University of Applied Sciences Utrecht}
\date{\today}

\newcommand{\columnblock}[2][Title]{
    \begin{block}{\large \vspace{-0.5cm}#1}
    #2
    \end{block}
}

\newcommand{\twocolumnsblock}[3][Title]{
    \columnblock[#1]{
        \begin{multicols}{2}
            #2
            \newpage
            #3
        \end{multicols}
    }
}

\begin{document}


\begin{frame}{}
    \columnblock[Introduction(1)]{
    \justifying 
Manufacturing paradigms such as Agile Manufacturing (AM) and Reconfigurable Machine Systems (RMS) revolve around flexibility. The REXOS project utilizes a Grid Manufacturing setup, which is based on RMS. In Grid Manufacturing, different approaches can be used for scheduling, namely hierarchical and heterarchical. This poster compares the two approaches using a simulation program. And explores the pros and cons considering both setups.
	} 
	
    \begin{columns}[t]   
		\begin{column}{.48\textwidth}
            \columnblock[image heter(2)]{
            \justifying 
			\includegraphics[trim= 15cm 0cm 15cm 0cm, clip=true, width=.98\linewidth]{GridEquipletHeter.jpg}\\
			In this setup a 
    }
                      
		\end{column}
		\begin{column}{.48\textwidth} 
 			\columnblock[image hier(3)]{
            \justifying 
            \includegraphics[trim= 15cm 0cm 15cm 0cm, clip=true, width=.98\linewidth]{GridEquipletHier.jpg}\\
            
    }
    
\end{column}
    \end{columns}
    
       \begin{columns}[t]  
			\begin{column}{.98\textwidth}
			\columnblock[intro to resulsts(6)]{
			text eHIER
				\justifying 
				}
            \end{column}
    \end{columns}
    
       \begin{columns}[t]  
			\begin{column}{.48\textwidth}
				\columnblock[Graphs(7)]{
			text en graphs hier
				\justifying 
				}
            \end{column}
			\begin{column}{.48\textwidth}
			\columnblock[Graphs(8)]{
			text en graphs hier
				\justifying 
				}
            \end{column}
    \end{columns}
    
   \begin{columns}[t]  
			\begin{column}{.98\textwidth}
				\columnblock[9]{
				\justifying 
%PRODUCTS IN GRID SIMULTANEOUSLY (1)
As seen in the graphs above, using heterarchical scheduling causes a higher production throughput within a certain timeframe. Hierarchical scheduling  on the other hand has a lower throughput due to the fact that the equiplets have a lower load.

%Throughput (2)
When using a hierarchical setup, products and batch products will be placed in a more controlled situation resulting in lower transportation needs. Heterarchical has higher transportation needs because products need to travel further to get to the equiplet which was chosen.

%FAILED Products (3)
A heterarchical setup tends to be more chaotic than a hierarchical setup. Production is faster, but products will stay in the grid for longer periods of time causing them to have a higher chance of failing. These products fail because their deadline has expired, which is the result of long transport times.
				}   

            \end{column}
    \end{columns}
    
   \begin{columns}[t]  
			\begin{column}{.65\textwidth}
							\columnblock[Conclusion(10)]{
							\justifying
Using a heterarchical scheduling setup, equiplets will have a higher throughput, but at the cost of transportation. Putting a high load on the grid could also result in more failed products. A heterarchical setup is recommended when there are a lot of varying products that need to be produced simultaneously.

Hierarchical scheduling is more efficient when there are limited transportation resources or if there is a need to control production. Reserving equiplets for batches of products could be desired to make sure the batch will be finished in time.
				}
            \end{column}
            \begin{column}{.30\textwidth}
               \columnblock[References(14)]{
				\justifying 
				Lorem ipsum dolor sit amet, consectetur adipiscing elit. Proin molestie tincidunt velit dictum auctor. Duis eu 		metus ac metus egestas venenatis. Lorem ipsum dolor 		sit amet, consectetur adipiscing elit. Mauris quis neque vel justo dapibus tincidunt. Vestibulum pellentesque bibendum tempor. Sed velit leo, dignissim vel lobortis non, 	aliquet id augue. Etiam vehicula lorem urna, ac condimentum ligula tempor id. Vestibulum rutrum lobortis odio. Vivamus ac porta arcu.
	}
            \end{column}
    \end{columns}
\end{frame}
\end{document}